
%% -----------------------------------------------------------------------------

\vspace{1.5em}
\textbf{How to Turn the Idea of the Rational Programmer into a Methodology.}
Every time the rational programmer succeeds, it is validation for programming
language researchers. It shows how their theorems, slogans, and tools help
programmers. The example of blame assignment mechanisms makes this point
clearly. A blame-assignment mechanism provides information that, according to
programming language research, points toward the source of the problem.

When the rational programmer fails, it questions programming
language research. Specifically, it indicates limited predictive
power of programming language theory with respect to the use of languages in
practice. Indeed, misleading predictions may even suggest flaws in language design.

In this way, the idea of a rational programmer implies an entire methodology for
evaluating the design of programming languages. At this point, this study of
methods supplies many questions whose answers might point to suitable evaluation
methods: 

\begin{enumerate} 

\item Does a language design provide information that can guide a
 rational programmer?

\item Does the underlying theory suggest actions to the rational programmer?

\item Can this guidance be formulated as an algorithm?

\item Does the underlying theory lead to a hypothesis about the effects of
 these actions? 

\item Can this hypothesis be tested with a large-scale automated experiment?

\end{enumerate}

Here is how the answers to these methodological questions lead to an evaluation
method for blame-assignment mechanisms: 

\begin{enumerate}

\item The design of blame-assignment mechanisms explicitly advertises the blame
information as helpful for debugging impedance mismatches.

\item The error messages of blame-assignment mechanisms include suspect
locations at the boundary of typed and untyped code fragments.
The Wadler--Findler slogan suggests that the source of the problem is concealed due to a lack of types, so
adding types to the untyped fragment should lead to the source of the impendance mismatch.

\item The step-by-step construction of paths based on error messages from
gradual-typing checks is clearly amenable to implementation, modulo the
ascription of types to modules.

\item The theory conjectures that blame assignment constrains the search
space that a developer must inspect to find the problem.

\item Based on these insights, the remaining sections detail a large-scale automated experiment.

\end{enumerate}
That said, using the method to conduct data-gathering experiments poses
several challenges. The specific challenges are spelled out in the next section,
and the following three sections explain ways of overcoming them.
