\section{How to Simulate the Rational Programmer} \label{sec:ideas}

Given the comparable implementations of the three gradual typing systems,
the next step for applying our method is to obtain a set of fully
typed benchmarks that each comes with a single impedance mismatch between
its type specifications and its code. As we briefly discuss in
section~\ref{sec:challenges}, we derive such ill-typed benchmarks with
mutation analysis from well-typed programs.  We return to mutation
analysis and the benchmarks in section~\ref{sec:experiment}.  Herein,
we assume that we have them in hand and instead we focus on the
main element of our method, the rational programmer.  


The rational programmer exploits blame information to figure out wih. Therefore the way to understand its behavior 
is to analyze the migratory path it follows through the 
\emph{configuration space} of the program. The space
consists of variants of the program, dubbed \emph{configurations}, that 
differ in their type annotations. 
Herein we build on top of Typed Racket's migratory typing
approach,\footnote{The method can be adapted to so-called true gradual typing
systems~\cite{svcb-snapl-2015} in a straightforward manner.}, hence, each configuration differs from another
in which of its components are typed and which are not. Formally, we can
represent a program $\system$ as a set of of components
$\set{\component}$ and each configuration $\conf$ of $\system$ is a subset of 
$\set{\component}$
that contains exactly the components of  $\system$ that are typed in
$\conf$. Naturally the configurations of  $\system$ are ordered by the subset relation and
thus form a lattice $\lattice{\system}$ with size $2^{\size{\system}}$.
The bottom configuration of $\lattice{\system}$ is always $\emptyset$ and
the top element is $\system$ itself. In between there are all the
mixed-type configurations. If we assume that  component $\component$ of $\system$
contains the type error, any configuration $\conf$ that
includes $\component$ is ill-typed. Any other configuration $\conf$ may blame 
a $\component_b$, which we denote with $\blame{\conf}$. Let a blametrail
be a sequence of configurations $\conf_0,...\conf_n$ where 
for all $0 \leq i \leq n - 1$, $\conf_i \subset \conf_{i+1}$ and
$\conf_{i+1} \setminus \conf_i = \blame{\conf_i}$. That is a blametrail
corresponds to a succession of a configurations that start with a
configuration $\conf_0$, dubbed the \emph{root of the blametrail}, and 
each configuration is derived from its previous one by typing the
component that the latter blames. A rational programmer is one that
migrates the variant of a program $\system$ that corresponds to
configuration $\conf_0$ along the blametrail that starts at $\conf_0$.
We say a blametrail is final iff its last configuration, $\conf_n$
does not blame a component, $\blame{\conf_i} = \emptyset$. We call a
blametrail \emph{successful} iff it is final and its last configuration
$\conf_n$ contains the faulty component of $\system$, $\faulty{\system}
\in \conf_n$. 

Using the notion of successful blametrails we can define precisely what it
means for blame to add value to a gradual typing system: 

All blametrails are successfull: Given a program P and its configuration
lattice $\lattice{\system}$, all blametrails in the lattice are successful. 

With these definitions we can describe with precision the experimental
process of our method. For each benchmark we create its configuration
lattice. Then we separate the configurations of the lattice that 
when run result in blame or result in a static type error from  those that do not
do either. The former are 
suitable to be roots of a blametrail. Thus we simulate the rational 
programmer and if the configuration blames we type their blamed component to obtain the next
configuration in the blametrail and repeat the process until the
blametrail becomes final. At this point we check whether the last
configuration fails to type check. If it does for all blametrails in the
lattice, then the property holds. Otherwise it fails. 

In addition to the property the framework allows us to assign a metric of
effort to a successful blametrail; $\effort{\conf_0, ...., \conf_n} = n$ 

\subsection{How to Generalize the Rational Programmer} 

As we discuss in section~\ref{sec:challenges}, an important factor that the method need to
take into account is that different gradual typing systems blame
differently and perform different checks. For that reason in
section~\ref{sec:challenges} we introduce the notion of modes for the
rational programmer. Defining these modes requires a generalization of the 
discussion above. 

In particular it requires revisiting the notion of blametrail and that of
successful blametrail. 

We define a blametrail in a mode  $\modem$ as  
a sequence of configurations $\conf_0,...\conf_n$ where 
for all $0 \leq i \leq n - 1$, $\conf_i \subset \conf_{i+1}$ and
$\conf_{i+1} \setminus \conf_i = \mode{\conf_i}$.

   For example we can obtain the simple blame mode
for the Natural system if we use, as above, the selector $\blame{\cdot}$. 
However, the same mode has a different meaning in Transient as this
semantics results in multiple blamed components. 

This requires generalizing the notion of successful blametrail
to that of successful blametrees: a blametree is the set of all
blametrails with root a configuration $\conf_0$. A blametree is succesful if at least one
of its blametrails is succesful. 

Of course for systems such as Transient we can add modes where the rational
programmer picks one blametrail as part of the migration process. For
example the \first{\blame{\cdot}} mode continues with the oldest blame
entry in the blameset of a configuration, as we discuss in
section~\ref{sec:challenges}. 

Moreover, the \emph{vanilla} mode corresponds to ignoring blame and instead
using only information from the dynamic type error that a configuration
produces using the $\error{\cdot}$ selector. 

Finally, the \emph{null} mode uses selector $\random{\cdot}$ and it is
used by our method as the null hypothesis control experiment.

The following table collects all the modes and their selectors for the
three gradual typing systems we compare in this paper:


\begin{tabular}{c|c|c|c}
                    & {\bf Natural}  & {\bf Transient}        & {\bf Erasure} \\
   \hline 
  {\bf Simple Blame}   & \blame{\conf}  & \blame{\conf}          &               \\
 {\bf First Blame}  &                 & \first{\blame{\conf}}  &               \\
  {\bf Vanilla}       &  \error{\conf} & \error{\conf}          & \error{\conf} \\
  {\bf Null}        & \random{\conf} & \random{\conf}         & \random{\conf} 
\end{tabular}  
  
 
As a last step we also need to lift the notion of effort to blametrees.
The effort of a blametree $\effort{\{\conf_0, ...., \conf_n\},...}$ is the
aggregate effort of all its blametrails.
































































































































