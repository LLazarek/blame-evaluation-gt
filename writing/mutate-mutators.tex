% -----------------------------------------------------------------------------

A {\it mutator\/} performs a localized syntactic change to a code base. The
result is a {\em mutant\/}.

For the evaluation of a blame strategy, mutators must produce type-level
mistakes that the run-time checks of gradual typing systems or the safety
checks of the underlying language can detect. Once detected, the rational
programmer should be able to locate the mistake by gradually adding types
to blamed modules. In other words, the suitability of the mutators hinges
on their ability to generate interesting debugging scenarios (see
sec.~\ref{sub:mutate-interesting}).

%% MF: I don't see what this contributes:
% \footnote{Many of~\citet{lksfd-popl-2020} mutators do not satisfy the
% constraints and had to be eliminated or significantly adapted. For example, the
% relational mutator does not lead to type-level errors. Similarly, the
% arithmetic mutator does not reliably lead to type errors; it had to be adapted
% to make only operator swaps that can affect the type of the result, e.g.,
% changing `*` to `/`.}

Figure~\ref{table:mutation-ops} describes 16 mutators that satisfy these
constraints. As the last column indicates, some specialize or
generalize~\citet{lksfd-popl-2020}'s mutators, which in turn are borrowed from
the vast literature on mutation testing~\citep{jia2011analysis}.  Only two are
directly inherited; many mutators are brand new. For the latter, the authors
relied on their decades-long experience of making type-level mistakes in Typed
Racket, some of which take non-trivial effort to debug.

\begin{wrapfigure}{r}{.50\textwidth} \footnotesize
\hspace{0.2cm}
\begin{module}{}\typecolor
(: int-f  (-> Integer Void))
(: bool-f (-> Boolean Void))
(: x      (U Integer Boolean))
(if (integer? x)
  (int-f x)
  (bool-f x))
\end{module}
\caption{Example program using occurrence typing}
\label{fig:negate-cond-example}
\end{wrapfigure}

Most of the mutators are self-explanatory.  The first six apply to all
gradually typed languages; the next six to those that include classes and
objects. The last four target distinguishing features of Typed
Racket's type system,
specifically its sophisticated type system. For example, one mutation produced by {\tt arithmetic}
replaces a {\tt +} with a {\tt -} in an attempt to change the type of the
arithmetic expression; {\tt +}'s result is a {\tt Positive-Integer} when all
arguments are positive integers, while {\tt -} yields {\tt
Integer}~\citep{stathff-padl-12}. Similarly, the other three aim to confound
the occurrence type system. For example, the {\tt negate-cond} mutator can
create interesting mistakes in programs like figure~\ref{fig:negate-cond-example}.

