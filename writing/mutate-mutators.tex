% -----------------------------------------------------------------------------

A {\it mutator\/} performs a localized syntactic change to a code base. The
result is a {\em mutant\/}.

For the evaluation of a blame strategy, mutators must produce type-level
mistakes that the run-time checks of gradual typing systems or the safety
checks of the underlying language can detect. Once detected, the rational
programmer should be able to locate the mistake by gradually adding types to
blamed modules.

Figure~\ref{table:mutation-ops} describes 16 mutators that satisfy these
constraints. As the explanation indicates, some specialize or
generalize~\citet{lksfd-popl-2020}'s mutators, which in turn are borrowed from
the vast literature on mutation testing~\cite{jia2011analysis}.\footnote{Many
of~\citet{lksfd-popl-2020} mutators do not satisfy the constraints and had to
be eliminated. For example, the relational mutator does not lead to type-level
errors. Similarly, the arithmetic mutator does not reliably lead to type
errors; it had to be adapted to make only operator swaps that can affect the
type of the result, e.g., changing `*` to `/`.}  Only two are directly
inherited; many mutators are brand new. For the latter the authors relied on
their decades-long experience of making type-level mistakes in Typed Racket,
some of which take a non-trivial effort to debug.

Most of the mutators are self-explanatory.  The first six apply to all
gradually typed languages; the next six to those that include classes and
objects. The last four target distinguishing features of Typed Racket,
specifically is sophisticated type system. In particular, {\tt arithmetic}
replaces a {\tt +} with a {\tt -} in an attempt to change the type of the
arithmetic expression; {\tt +}'s result is a {\tt Positive-Integer} when all
arguments are positive integers, while {\tt -} yields {\tt
Integer}~\cite{stathff-padl-12}. Similarly, the other three also aim to usurp
Typed Racket's occurrence type system.

