% -----------------------------------------------------------------------------

A {\it mutator\/} performs a localized syntactic change to a code base. The
result is a {\em mutant\/}.

For the evaluation of a blame strategy, the mutator must produce type-level
mistakes that the run-time checks of gradual typing systems or the safety
checks of the underlying language can detect. Once detected, the rational
programmer should be able to detect the mistake by gradually adding types to
blamed modules.

Figure~\ref{table:mutation-ops} describes the 16 mutators used for the
experiment.  They are inspired by the authors decades-long experience
of making type-level mistakes in Typed Racket, some of which take a
non-trivial effort to debug.  For each mutaror, the figure provides a
short description and an example mutation.

Most mutators are self-explanatory.  The first twelve apply to most
gradually typed languages, including those with classes.  The last
four target distinguishing features of Typed Racket.  Specifically,
{\tt arithmetic} may replace a {\tt +} with a {\tt -} in an attempt to
change the type of the result of the arithmetic operation. In Typed
Racket, {\tt +}'s result is a {\tt Positive-Integer} when all
arguments are positive integers~\cite{stathff-padl-12}. However, the
result of {\tt -} is {\tt Integer}. In the same spirit, the {\tt
boolean} mutator aims to take advantage of Typed Racket faithfulness
to Racket's ``falsiness.'' Finally, {\tt negate-cond} and {\tt
force-cond} attempt to confuse occurrence typing.
