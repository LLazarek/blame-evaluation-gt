Of course, the gradual typing benchmarks have no mistakes for the rational
programmer to debug. Therefore, we follow ~\citet{lksfd-popl-2020} and we
inject bugs with mutation
analysis~\cite{jia2011analysis,demillo1978hints,lipton1971fault}.  A
mutation is a local syntactic change of the code that may
produce a bug. Usually, the outcome of mutating a program once, dubbed a
mutant, can be distinguished from the original program with a test case.
The better the test suite of a program, the more mutants it can ``kill.''
Mutants that are hard to ``kill'' usually correspond to subtle programming
errors, and mutation frameworks typically come with standard mutators
carefully designed to create such mutants.

Unlike the hard-to-kill mistakes introduced by standard mutators, 
gradual type systems catch shallow errors.  Thus, in order to
exercise the rational programmer in the context of gradual typing, 
we need to develop alternative mutators
that are fine-tuned to produce the type-level mistakes gradual type systems 
can detect. Our inspiration is our
more-than-a-decade long experience of making type-level mistakes in Typed
Racket. Some of these mistakes often take non-trivial effort to debug and
have become the seed for our mutators, which we summarize in
figure~\ref{table:mutation-ops}.  For each one, the figure provides a
short description and an example mutation.


Most of our mutators are self-explanatory.
The first twelve apply to most gradually typed languages, including those with classes.
The last four mutators target distinguishing features of Typed Racket.
Specifically, \texttt{arithmetic} may replace a \texttt{+} with a \texttt{-}
in an attempt to change the type of the result of the arithmetic
operation. In Typed Racket, \texttt{+}'s result is a
\texttt{Positive-Integer} when all arguments are
positive integers. However, the result of \texttt{-} is
\texttt{Integer}. In the same spirit, \texttt{boolean} aims to take
advantage of Typed Racket's ``truthiness.'' Finally, \texttt{negate-cond} and \texttt{force-cond}
attempt to confuse occurrence typing.
