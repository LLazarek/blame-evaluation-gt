%% -----------------------------------------------------------------------------

The Transient semantics assigns blame to a sequence of modules. The blame
assignment says that the value witnessing the impedance mismatch may have
crossed the boundaries between elements in the set, and each crossing
checked the value's type in a shallow manner.

This ambiguity in Transient blame raises the question of how the rational
programmer should react when the language produces a blame set. Our answer
is that the rational programmer has at least two reasonable options. The first
one is to select the module that is added to the blame sequence first and
assign types to only that one---after all, if fully checked, the types of this
first module should be able to detect an impedance mismatch earlier in the
evaluation of a program than the later ones. The second option is to select the
module that is added to the blame sequence last, effectively interpreting the
blame sequence as a boundary-aware stack.

These two modes of rationalizing give rise to two different notions of trail.

\begin{quote}
\it A \emph{Transient-first blame trail}/\emph{Transient-last blame trail} is a sequence of scenarios
$\conf_0,...\conf_n$ of $\system$ where for all $0 \leq i \leq n - 1$,
$\conf_i \subset \conf_{i+1}$ and
\[ \conf_{i+1} \setminus \conf_i = %
\left\{ %
\begin{tabular}{ll} %
$\{\first{\mblame{\system, \conf_i}}\}$  & if $\conf_i$ produces blame \\ %
$\{\exception{\system, \conf_i}\}$     & otherwise \\ %
\end{tabular} %
\]
where $\first{\mblame{\system, \conf}}$/$\last{\mblame{\system, \conf}}$ is the
first/last module, respectively,  that Transient adds to the blame sequence for $\conf$.
\end{quote}

The definition of \emph{Transient exception trails} is analogous to that for
Natural. It is used to as a baseline for Transient-first and Transient-last
blame.
