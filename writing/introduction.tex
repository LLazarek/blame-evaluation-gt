\section{Does Blame Matter?}

Since its very beginning [cite STH], the theory of gradual typing
comes with a blame theorem. ``Well-typed [components]\footnote{The
original authors got this word wrong, and more.} can't go wrong'' has
become the slogan. The idea is that blame assignment can help the
practicing programmers find impedance mismatches between the types
they added and the remaining untyped components.

Practical gradual type systems ignore blame completely. Industrial
systems [lots of citations. footnotes] exploit types for assistance
with IDE actions and for finding typos in code. Then they remove them
and rely on the built-in safety checks of the underlying programming
language to catch problems.

This contrast raises the question whether

     blame adds value to gradual typing
     and especially the working programmer.

To obtain an answer, we have developed a method (not methodology!)  to
evaluate blame assignment systems from the gradual typing
world. Roughly speaking, the method simulates a programmer who follows
the above-cited Wadler-Findler slogan. When the run-time system
signals a violation, the programmer conjectures that the addition of 
types to untyped components turns the dynamic violation into a static
type error and thus reveals the impedance mismatch.

The paper presents the evaluation method and the results of applying
the method to three distinct blame assignment systems.

- explain the systems
- explain the method
- preview the results

- weave roadmap into these paragraphs 

