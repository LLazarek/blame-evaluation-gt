\section{Does Blame Matter?}
\label{sec:introduction}

Theoreticians of gradual typing have focused on
blame theorems from the very beginning~\cite{mf-toplas-2009, tf-dls-2006}. ``Well-typed
[components]\footnote{The original authors got this word wrong.} can't be
blamed'' turned the theorem into a slogan~\cite{wf-esop-2009}. Indeed,
academic implementations---for example, Reticulated
Python~\cite{vsc-dls-2019, vss-popl-2017, vksb-dls-2014} and Typed
Racket~\cite{tf-dls-2006,tf-popl-2008,tfffgksst-snapl-2017,tf-icfp-2010}---come
with sophisticated blame assignment strategies. The academics that
created them firmly embrace the idea that blame can help practicing
programmers find impedance mismatches between the types they added to a
software system and the behavior of the remaining untyped components. 

Industrial implementors of  gradual typing systems have almost completely ignored blame assignment.
Systems such as Flow,\footnote{\url{https://flow.org}}
Hack,\footnote{\url{https://hacklang.org}} or
TypeScript\footnote{\url{https://www.typescriptlang.org}} exploit types for IDE actions and for finding typos in
code. Then their compilers remove types and rely on the built-in safety checks of
the underlying language to catch any problems. 
Recently, \citet{cc-snapl-19} have turned this implicitly expressed
skepticism into an explicit statement and challenge.

This contrast between theoretical research and industrial applications of gradual typing raises the question 
\begin{quote}
 \it
 whether blame assignment adds any value to a gradually typed language,
 especially for the benefit of the working programmer.
\end{quote}
Given the long standing academic interest in blame  and its complete absence in
industrial systems, it comes as an even bigger surprise that the research
literature and the industrial blog world do not present or discuss any
possible answers.  Instead, when language designers decide to look into the
question, they seem to answer it one way or another without any scientific
justification. As a matter of fact, the research community simply does not have a
method for evaluating blame, and given the question's practical
relevance, this is a significant problem.

This paper's first contribution is an automated method for evaluating the
effectiveness of blame assignment strategies in the gradual typing world.
%% (sections ~\ref{sec:landscape},~\ref{sec:mutate},~\ref{sec:rational} and \ref{sec:sample}). 
%% MF: says you can't simulate a roadmap by listing all sections.
%% MF: and for heaven's sake lookup the use of ~ in TeX
Roughly speaking, the method, which is
loosely based on the work of Lazarek et al.~\cite{lksfd-popl-2020},
simulates a programmer who follows the Wadler-Findler slogan quoted above.
Such a \emph{rational programmer} can find impedance-mismatch
bugs by adding types to a partially typed program (section~\ref{sec:landscape}). That is, when the
run-time system signals an impedance mismatch, a good blame message helps
the programmer infer where to add an additional type specification.
Components that pass the type checker cannot cause impedance mismatches,
so if the type checker discovers these mismatches at compile time then it points
directly to the faulty component. Otherwise, if the
additional type does {\em not\/} cause a static type error, the program
is run again to signal another violation. Then the programmer can heed
blame one more time to add yet another type specification.
Weighing whether we discover the mistake in this manner, and the number of
steps it takes to do so, are measures of the effectiveness
of the blame assignment strategy of the language.

The paper's second contribution is a set of results from applying the evaluation
method to three distinct blame assignment strategies and approximately
722,000 cases (section~\ref{sec:results}): Vitousek et
al.'s strategy of tracking typed/untyped boundaries in Reticulated, called Transient;
Tobin-Hochstadt et al.'s use of higher-order contract system~\cite{ff-icfp-2002,
mf-toplas-2009}, called Natural; and the approach of industrial systems that forego blame
and instead rely on error messages from the
underlying programming language, called Erasure. A direct interpretation of the results
(section~\ref{sec:discussion}) is that
they validate the conjecture behind the work of theoreticians. A good blame assignment
strategy helps the search for impedance mismatches between the specified types
and the behavior of untyped components.
Furthermore, the evaluation method
identifies the wrapper-based blame tracking of Typed Racket as the most
useful approach, with that of Reticulated Python close behind.

Next, the paper describes the challenges of developing an evaluation
method for blame, which may explain why the question has been neglected for such
a long time.


