%% ------------------------------------------------------------------------------

Theoreticians of gradual typing have focused on blame theorems from the
very beginning~\cite{mf-toplas-2009, tf-dls-2006}. ``Well-typed
[components]\footnote{The original authors got this word wrong.} can't be
blamed'' turned the theorem into a slogan~\cite{wf-esop-2009}. Academic
systems (Reticulated Python~\cite{vsc-dls-2019, vss-popl-2017,
vksb-dls-2014} and Typed
Racket~\cite{tf-dls-2006,tf-popl-2008,tfffgksst-snapl-2017,tf-icfp-2010})
come with sophisticated checking and blame assignment strategies (see
sec.~\ref{sec:landscape}). Their academic creators embrace the idea that
blame can help practicing programmers find impedance mismatches, that is,
disagreements between the type ascriptions of a software system and its
behavior.

Industrial implementors of gradual typing systems have almost
completely ignored blame assignment.  Systems such as Flow, Hack, or
TypeScript\footnote{See \url{https://flow.org},
\url{https://hacklang.org}, and \url{https://www.typescriptlang.org},
respectively.} exploit types for IDE actions and for finding typos in
code. Then their compilers remove types and rely on the built-in
safety checks of the underlying language to catch any problems.

This contrast between theory and applications of gradual typing raises the question of
\begin{quote}
 \it
 whether blame assignment adds any value to a gradually typed language,
 especially for the benefit of the working programmer.
\end{quote}
Given the long-standing academic interest in blame and its complete absence in
industrial systems, it comes as an even bigger surprise that the research
literature and the industrial blog world do not discuss any possible answers.
Instead, when language designers make decisions concerning this aspect, they seem
to go one way or another without any scientific justification. Then again, the
community has thus far failed to offer a method for evaluating blame assignment.

This paper's {\em first contribution\/} is {\em a method for evaluating the
effectiveness of blame assignment strategies\/} in the gradual typing world.
The top-level innovation is the idea of a {\em rational programmer\/}, that is,
a programmer that acts only in response to available information (see
sec.~\ref{sec:why-rational}). In the case of an impedance mismatch, the
available information consists of the error message and the current state of the
program. The rational programmer can hence use the former to change the
latter---and this systematic, information-driven process can be implemented and
tested, at scale, on real programs.  Turning this idea into a scientific
experiment requires overcoming major challenges: injecting representative
impedance mismatches; putting the various kinds of error information to
comparable use; and sampling the huge space of possibilities (see
sec.~\ref{sec:challenges} and, for details, secs.~\ref{sec:rational} through~\ref{sec:sample}).

The paper's {\it second contribution\/} is a set of {\em results from
applying the evaluation method to three distinct blame assignment
strategies in approximately 72,000 different scenarios\/} (see
sec.~\ref{sec:results}): (1)~{\it Transient\/}, i.e. Reticulated's
tracking of typed/untyped boundaries~\cite{vss-popl-2017}; (2)~{\it
Natural\/}, i.e. Typed Racket's use of a higher-order contract system and
its blame assignment~\cite{ff-icfp-2002}; and (3) {\it Erasure\/}, i.e.
the approach of industrial systems, which forgo blame in favor of error
messages from the safety checks of the underlying language. The results
(see sec.~\ref{sec:discussion}) are at least somewhat surprising.  In
principle they validate the conjectures behind the work of theoreticians.
First, a good blame assignment strategy helps with the search for
impedance mismatches between the specified types and the behavior of
untyped components.  Second, Natural's wrapper-based blame tracking is
more useful than Transient's ``collective blame'' tracking algorithm,
which in turn is superior to Erasure. {\em But\/}, the application of the
method also indicates problems with the expectations of theoreticians. The
first problem concerns the difference between the methods; neither is {\it
Natural\/} vastly more useful than {\it Transient\/} nor are the two
blame assignment methods highly superior to {\it Erasure\/}.  The second
one is that the cost of Transient's blame can be huge.  In turn, these
problems suggest that, on one hand, the existing theory does not predict
practice properly, and on the other hand, the existing practice may need
additional experiments to collect more data.
