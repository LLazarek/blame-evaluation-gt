
Theoreticians of gradual typing have focused on blame theorems from the very
beginning~\cite{mf-toplas-2009, tf-dls-2006}. ``Well-typed
[components]\footnote{The original authors got this word wrong.} can't be
blamed'' turned the theorem into a slogan~\cite{wf-esop-2009}. Academic systems
(Reticulated Python~\cite{vsc-dls-2019, vss-popl-2017, vksb-dls-2014} and Typed
Racket~\cite{tf-dls-2006,tf-popl-2008,tfffgksst-snapl-2017,tf-icfp-2010}) come
with sophisticated blame assignment strategies. Their academic creators embrace
the idea that blame can help practicing programmers find impedance mismatches
between the types they added to a software system and the behavior of the
remaining untyped components.

Industrial implementors of gradual typing systems have almost completely ignored
blame assignment.  Systems such as Flow,\footnote{\url{https://flow.org}}
Hack,\footnote{\url{https://hacklang.org}} or
TypeScript\footnote{\url{https://www.typescriptlang.org}} exploit types for IDE
actions and for finding typos in code. Then their compilers remove types and
rely on the built-in safety checks of the underlying language to catch any
problems.

This contrast between theoretical research and industrial applications of gradual typing raises the question 
\begin{quote}
 \it
 whether blame assignment adds any value to a gradually typed language,
 especially for the benefit of the working programmer.
\end{quote}
Given the long-standing academic interest in blame and its complete absence in
industrial systems, it comes as an even bigger surprise that the research
literature and the industrial blog world do not discuss any possible answers.
Instead, when language designers make relevant decisions, they seem to answer it
one way or another without any scientific justification. Then again, the
community has thus far failed to offer a method for evaluating blame assignment.

This paper's first contribution is {\em an automated method for evaluating the
effectiveness of blame assignment strategies\/} in the gradual typing world.
speaking, the method simulates a programmer who follows the Wadler-Findler
slogan quoted above; its automation borrows elements of of Lazarek et
al.~\cite{lksfd-popl-2020}'s recent work.\footnote{\citet{cc-snapl-19} use this
tracing method to localize type errors reported by gradual typing systems based
on inference.}  Such a \emph{rational programmer} can find impedance-mismatch
bugs by adding types to a partially typed program (see
section~\ref{sec:landscape}). That is, when the run-time system signals an
impedance mismatch, a good blame message helps the programmer infer where to add
an additional type specification because, by the basic theory, well-typed
components cannot cause impedance mismatches. Otherwise, if the additional type
does {\em not\/} cause a static type error, the program is run again to signal
another violation. Then the programmer can heed blame one more time to add yet
another type specification. Evaluating the quality of blame assignment thus
becomes a question of whether the tracing the error messages finds the source of
the mistake and how many steps it takes to get to this point.

The paper's second contribution is a set of {\em results from applying the
evaluation method to three distinct blame assignment strategies and
approximately 72,200 cases\/} (see section~\ref{sec:results}): (1) {\it
Transient\/}, i.e. Reticulated's tracking of typed/untyped boundaries; (2) {\it
Natural\/}, i.e. the use of higher-order contract system and its blame
assignment~\cite{ff-icfp-2002}; and (3) {\it Erasure\/}, i.e. the approach of
industrial systems, which forego blame assignment and rely on error messages
from the safety checks of the underlying programming language.

The results of the method's first application (see section~\ref{sec:discussion})
are at least somewhat surprising. In principle they validate the conjectures
behind the work of theoreticians. First, a good blame assignment strategy helps
with the search for impedance mismatches between the specified types and the
behavior of untyped components.  Second, Natural's wrapper-based blame tracking
is more useful than Transient's ``collective blame'' tracking algorithm, which
in turn is superior to Erasure. {\em But\/}, the application of the method also
indicates problems with the expectations of theoreticians. The first problem is
that the cost of Transient's blame can be huge. The second one concerns the
difference between the methods; neither is {\it Natural\/} vastly better than
{\it Transient\/} nor are the two blame assignment methods clearly preferable to
{\it Erasure\/}. The third one casts doubt on the current the trade-off between
the cost of blame-assignment systems and the quality of their blame messages.
In turn, these problems suggests that, on one hand, the existing theory does not
predict practice properly, and on the other hand, the existing practice needs to
be supported by additional data. 




