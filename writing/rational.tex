%% -----------------------------------------------------------------------------

Section~\ref{sec:why-rational} explains how a migratory type setting helps with
finding the source of an impedance mismatch. Roughly speaking, it encourages the
rational programmer to equip a module with types if it is blamed in an error
message. A sequence of such steps makes up a path in the
lattice of type migration~\citep{tfgnvf-popl-2016}. The lattice describes the
space in which the modes of the rational programmer search for bugs (see
sec.~\ref{sub:stuff}).
%\mf{what does this sentence really mean? what does it contribute?}
%\ll{Addressed}

Each mode receives different kinds of information and thus may construct
different paths in the lattice. As discussed in section~\ref{sec:challenges},
evaluating blame relies on comparing modes of the rational programmer
within the same semantics and across different semantics.  
Hence the research problem is to develop modes for the rational programmer and to make
them comparable even when they correspond to different semantics and process
different kinds of information (see secs.~\ref{sub:natural}
through~\ref{sub:erasure}). Programmer effort relative to a fixed debugging
scenario introduces another dimension along which the modes become may be
compared (see sec.~\ref{sub:effort}). With these notions in place, it
becomes possible to state the experimental questions and describes the process (see sec.~\ref{sub:experiment}).

%% -----------------------------------------------------------------------------
\def\rsub#1#2{\subsection{#2} \label{sub:#1} \input{rational-#1.tex}}

\rsub{stuff}     {The Lattice and the Debugging Scenario} 
\rsub{natural}   {The Natural Rational Programmer} 
\rsub{transient} {The Transient Rational Programmer} 
\rsub{erasure}   {The Erasure Rational Programmer} 
\rsub{effort}    {The Programmer Effort} \label{subsec:effort}
\rsub{experiment}{The Experimental Questions} 
