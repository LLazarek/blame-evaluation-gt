%% -----------------------------------------------------------------------------

Section~\ref{sec:why-rational} explains, and illustrates with an example,
how a migratory type setting helps with finding the source of an impedance
mismatch. Roughly speaking, it encourages the rational programmer to
equip a module with types if it is blamed in an error message. A sequence
of such steps make up a portion of the migratory path in the lattice of
type migration~\cite{tfgnvf-popl-2016}. The lattice, described in Section
~\ref{sub:stuff},  serves as the common substrate for the definition modes
of the rational programmer mentioned in section~\ref{sec:challenges}. 

Each mode receives different kinds of information and thus may construct
different paths in the lattice. As discussed in section~\ref{sec:challenges},
evaluating blame relies on comparing modes of the rational programmer
within the same semantics and across different semantics.  

Hence the research problem is how to make the modes of the rational programmer
comparable even when they correspond to different semantics and offer different kinds of information. Sections~\ref{sub:natural}
through~\ref{sub:erasure} describe the various modes of the rational programmer and explain
the extent to which they are comparable. Section~\ref{sub:experiment}
enhances their comparability with the programmer-effort dimension. Finally, the last
subsection states the experimental questions and gives an overview of the
experimental process. 

%% -----------------------------------------------------------------------------
\def\rsub#1#2{\subsection{#2} \label{sub:#1} \input{rational-#1.tex}}

\rsub{stuff}     {The Lattice and the Debugging Scenario} 
\rsub{natural}   {The Natural Rational Programmer} 
\rsub{transient} {The Transient Rational Programmer} 
\rsub{erasure}   {The Erasure Rational Programmer} 
\rsub{effort}    {The Programmer Effort} \label{subsec:effort}
\rsub{experiment}{The Experimental Questions} 
