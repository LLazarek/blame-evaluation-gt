%% -----------------------------------------------------------------------------

Section~\ref{sec:why-rational} explains, and illustrates with an example, how a
migratory type setting helps find the source of an impedance mismatch. Roughly
speaking, it encourages the rational programmer equip a module with types if it
is blamed in an error message. A sequence of such steps make up a portion of the
migratory path in the lattice of type migration~\cite{tfgnvf-popl-2016} (see
sec.~\ref{sub:stuff} for details). 

Unfortunately, the rational programmer receives different kinds of information
and thus may construct different paths. As section~\ref{sec:landscape} shows,
the three different semantics deal with the same program in three different
ways, and the error messages point to different sources of a problem.

Hence the research problem is how to view the actions of a rational programmer
comparable across all semantics. Sections~\ref{sub:natural}
through~\ref{sub:erasure} describe the various rational programmers and explain
to what extent their actions are comparable.

%% -----------------------------------------------------------------------------
\def\rsub#1#2{\subsection{#2} \label{sub:#1} \input{rational-#1.tex}}

\rsub{stuff}     {The Lattice and the Debugging Scenario} 

\rsub{natural}   {The Natural Rational Programmer} 
\rsub{transient} {The Transient Rational Programmer} 
\rsub{erasure}   {The Erasure Rational Programmer} 
\rsub{effort}    {Programmer Effort} \label{subsec:effort}
