%% -----------------------------------------------------------------------------

Section~\ref{sec:why-rational} explains how a migratory type setting helps with
finding the source of an impedance mismatch. Roughly speaking, it encourages the
rational programmer to equip a module with types if it is blamed in an error
message. A sequence of such steps make up a portion of the migratory path in the
lattice of type migration~\cite{tfgnvf-popl-2016}. The lattice serves as the
common substrate for the definition modes of the rational programmer (see
sec.~\ref{sub:stuff}).

Each mode receives different kinds of information and thus may construct
different paths in the lattice. As discussed in section~\ref{sec:challenges},
evaluating blame relies on comparing modes of the rational programmer
within the same semantics and across different semantics.  

Hence the research problem is develop modes for the rational programmer and to make
them comparable even when they correspond to different semantics and offer
different kinds of information (see secs.~\ref{sub:natural}
through~\ref{sub:erasure}). Programmer effort relative to a fixed debugging
scenario introduces another dimension along which the modes become may be
compared (see sec.~\ref{sub:effort}). With these notions in place, it
becomes possible to state the experimental questions and gives an overview of
the experimental process (see sec.~\ref{sub:experiment}).

%% -----------------------------------------------------------------------------
\def\rsub#1#2{\subsection{#2} \label{sub:#1} \input{rational-#1.tex}}

\rsub{stuff}     {The Lattice and the Debugging Scenario} 
\rsub{natural}   {The Natural Rational Programmer} 
\rsub{transient} {The Transient Rational Programmer} 
\rsub{erasure}   {The Erasure Rational Programmer} 
\rsub{effort}    {The Programmer Effort} \label{subsec:effort}
\rsub{experiment}{The Experimental Questions} 
