%% -----------------------------------------------------------------------------

Putting a rational programmer to work means generating many mutants and turning
those into debugging scenarios. Since existing mutators do not generate useful
mutants, the first step is to develop new mutators (see sec.~\ref{sub:mutate-mutators})
and to validate their suitability for this application (see sec.~\ref{sub:mutate-interesting}).

% As section~\ref{sec:challenges} explains, the first challenge is to
% generate a representative mistakes in representative software. This
% section explains the specific choices made for the first application
% of the method: the source and nature of the chosen benchmarks
% (sec.~\ref{sub:mutate-benchmarks}), the newly developed mutators
% (sec.~\ref{sub:mutate-mutators}), and why they are suitable for this
% experiment (sec.~\ref{sub:mutate-interesting}). 

%% -----------------------------------------------------------------------------
\def\sub#1#2{\subsection{#2} \label{sub:mutate-#1} \input{mutate-#1.tex}}


\begin{figure*} \footnotesize
\begin{tabular}{p{1.5cm} | p{6.9cm} | p{1.9cm} | r | r}
             {\bf  name} & {\bf description} & {\bf author} & {\bf loc} & {\bf mod.} \\ \hline
%% ---------------------------------------------------------------------------------------------------------------

\texttt{acquire} & object-oriented board game implementation      & M. Felleisen & 1941 & 9 \\ \hline

\texttt{gregor}  & utilities for calendar dates                   & J. Zeppieri  & 2336 & 13\\ \hline

\texttt{kcfa}    & functional implementation of 2CFA for \(\lambda\) calculus       & M. Might & 328 & 7\\ \hline

\texttt{quadT}   & converter from S-expression source code to PDF & M. Butterick & 7396 & 14\\ \hline

\texttt{quadU}   & converter from S-expression source code to PDF & B. Greenman  & 7282 & 14 \\ \hline

\texttt{snake}   & functional implementation of the Snake game    & D. Van Horn & 182 & 8 \\ \hline

\texttt{synth}  & converter of notes and drum beats to WAV & V. St-Amour & 871 & 10 \\ \hline

\texttt{take5} & mixin-based card game simulator & M.Felleisen & 465 & 8\\ \hline

\texttt{tetris} & functional implementation of Tetris & D. Van Horn &   280 & 9 \\ \hline

\texttt{suffix\-tree} & algorithm for common longest subsequences between strings & D. Yoo & 1500 & 6 \\

\end{tabular}

\caption{Summary of benchmarks} \label{table:benchmark-descriptions}

\end{figure*}


\sub{benchmarks}{The Experimental Benchmarks}
%% \example{ from text } -> {to text} 
\long\def\example#1 -> #2{
  \begin{tabular}[t]{@{}l}
#1 $\rightarrow$ #2 
\end{tabular}}

\long\def\examplesplit#1 -> #2{
  \begin{tabular}[t]{@{}l}
#1\\
$\rightarrow$  #2 
\end{tabular}}


\def\origin#1{%
  #1\\
  \hline}

\def\originspecial{
\origin{specialization of previous mutator~\cite{lksfd-popl-2020}}}

\def\origingen{
\origin{generilization of previous mutator~\cite{lksfd-popl-2020}}}

\def\originprevious{
\origin{previous mutator~\cite{lksfd-popl-2020}}}


\def\originnew{
\origin{new}}


\def\classpr#1 \\
  \quad {\tt (define/#1 (m x)} \\
   \quad \quad  {\tt x))}
\end{tabular}
\end{minipage}}


\def\classsuperclass#1{%
\begin{minipage}[t]{3.5cm}
  \begin{tabular}[t]{@{}l}
  {\tt (class #1} \\
   \quad {\tt (super-new))} \\
\end{tabular}
\end{minipage}}


\def\classsupernew#1 \\
   \quad {\tt #1)} \\
\end{tabular}
\end{minipage}}



\begin{figure*}
  \begin{tabular}{@{}p{2.1cm}@{\,\,}|@{\,\,}p{3.5cm}@{\,\,}|@{\,\,}p{5cm}@{\,\,}|@{\,\,}p{2.3cm}@{} }
    {\bf name} & {\bf description} & {\bf example} & {\bf origin}\\ \hline
%% -----------------------------------------------------------------------------
{\tt constant}
 & Swaps a constant with another of the same value but different type.
 & \example{5.6} -> {5.6+0.0i} 
 & \originspecial 

{\tt deletion}
 & Deletes the result expression of a sequence.
 & \examplesplit{{\tt (begin x y z)}} -> {{\tt (begin x y)}} 
 & \originspecial 

{\tt position}
  & Swaps two sub-expressions.
  & \examplesplit{{\tt (f a 42 "b" 0)}} -> {{\tt (f a 42 0 "b")}}
  & \origingen 

{\tt list}
 & Replaces {\tt append} with {\tt cons}.
 & \example{{\tt append}} -> {{\tt cons}} 
 & \originnew 

{\tt top-level-id}
 & Swaps identifiers defined in the same module.
 & \example{{\tt (f x 42)}} -> {{\tt (g x 42)}} 
 & \originnew  

{\tt imported-id}
 & Swaps identifiers imported from the same module.
 & \example{{\tt (f x 42)}} -> {{\tt (g x 42)}} 
 & \originnew   

{\tt method-id}
 & Swaps two method identifiers.
 & \examplesplit{{\tt (send o f x 42)}} -> {{\tt (send o g x 42)}} 
 & \originnew   


{\tt field-id}
 & Swaps two field identifiers.
 & \examplesplit{{\tt (get-field o f)}} -> {{\tt (get-field o g)}} 
 & \originnew   
  

{\tt class:init}
 & Swaps values of class initializers.
 & \examplesplit{{\tt (new c [a 5] [b "hello"])}} -> {{\tt (new c [a "hello"] [b 5])}} 
 & \originnew   

{\tt class:parent}
 & Replaces the parent of classes with {\tt object\%}.
    & \examplesplit{\classsuperclass{foo\%}} -> {\classsuperclass{object\%}}
 &  \originnew    
 

{\tt class:public}
 & Makes a public method private and vice versa.
 & \examplesplit{\classpr{public}} -> {\classpr{private}}
 & \originnew  

{\tt class:super}
 & Removes {\tt super-new} calls from class definitions.
 & \examplesplit{\classsupernew{(super-new)}} -> {\classsupernew{(void)}} 
 & \originnew

{\tt arithmetic}
 & Swaps arithmetic operators.
 & \example{{\tt +}} -> {{\tt -}}
 & \origingen

{\tt boolean}
 & Swaps {\tt and} and {\tt or}.
 & \example{{\tt and}} -> {{\tt or}}
 & \originprevious

{\tt negate-cond}
 & Negates conditional test expressions.
 & \examplesplit{{\tt (if (= x 0) t e)}} -> {{\tt (if (not (= x 0)) t e)}}
 & \originprevious

{\tt force-cond}
 & Replaces conditional test expressions with {\tt \#t}.
 & \examplesplit{{\tt (if (= x 0) t e)}} -> {{\tt (if \#t t e)}}
 & \originnew

\end{tabular}

\caption{Summary of mutators.} \label{table:mutation-ops}
\end{figure*}

\sub{mutators}   {How to Mutate Software} 
\sub{interesting}{Are These Mutators Interesting} 
