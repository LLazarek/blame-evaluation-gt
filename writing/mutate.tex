%% -----------------------------------------------------------------------------

As section~\ref{sec:challenges} explains, the first challenge is to
generate a representative mistakes in representative software. This
section explains the specific choices made for the first application
of the method: the source and nature of the chosen benchmarks
(sec.~\ref{sub:mutate-benchmarks}), the newly developed mutators
(sec.~\ref{sub:mutate-mutators}), and why they are suitable for this
experiment (sec.~\ref{sub:mutate-interesting}). 

%% -----------------------------------------------------------------------------
\def\sub#1#2{\subsection{#2} \label{sub:mutate-#1} \input{mutate-#1.tex}}


\begin{table} \footnotesize
  \caption{Summary of benchmarks}
\begin{tabular}{p{1.5cm} | p{6.9cm} | p{1.9cm} | r | r}
             {\bf  name} & {\bf description} & {\bf author} & {\bf loc} & {\bf mod.} \\ \hline
%% ---------------------------------------------------------------------------------------------------------------

\texttt{acquire} & object-oriented board game implementation      & M. Felleisen & 1941 & 9 \\ \hline

\texttt{gregor}  & utilities for calendar dates                   & J. Zeppieri  & 2336 & 13\\ \hline

\texttt{kcfa}    & functional implementation of 2CFA for \(\lambda\) calculus       & M. Might & 328 & 7\\ \hline

\texttt{quadT}   & converter from S-expression source code to PDF & M. Butterick & 7396 & 14\\ \hline

\texttt{quadU}   & converter from S-expression source code to PDF & B. Greenman  & 7282 & 14 \\ \hline

\texttt{snake}   & functional implementation of the Snake game    & D. Van Horn & 182 & 8 \\ \hline

\texttt{synth}  & converter of notes and drum beats to WAV & V. St-Amour & 871 & 10 \\ \hline

\texttt{take5} & mixin-based card game simulator & M.Felleisen & 465 & 8\\ \hline

\texttt{tetris} & functional implementation of Tetris & D. Van Horn &   280 & 9 \\ \hline

\texttt{suffix\-tree} & algorithm for common longest subsequences between strings & D. Yoo & 1500 & 6 \\

\end{tabular}

\label{table:benchmark-descriptions}

\end{table}


\sub{benchmarks} {Which Software to Use As Benchmarks}
\sub{mutators}   {How to Mutate the Benchmarks} 
%% \example{ from text } -> {to text} 
\long\def\example#1 -> #2{
  \begin{tabular}[t]{@{}l}
#1 $\rightarrow$ #2 
\end{tabular}}

\def\origin#1{%
  #1\\
  \hline}

% specialization of previous mutator~\cite{lksfd-popl-2020}

\def\explanation{$\ddagger$ inherited from, $+$ specializes one of, $++$
generalizes one of \citet{lksfd-popl-2020}'s mutators}

\def\originspecial{\origin{$+$}}

% {generalization of previous mutator~\cite{lksfd-popl-2020}}

\def\origingen{\origin{$++$}}

% {previous mutator~\cite{lksfd-popl-2020}}
\def\originprevious{\origin{$\ddagger$}}

\def\originnew{\origin{new}}

\def\classpr#1 \\
  \quad {\tt (define/#1 (m x)} \\
   \quad \quad  {\tt x))}
\end{tabular}
\end{minipage}}

\def\classsuperclass#1#2{%
\begin{minipage}[t]{4.2cm}
  \begin{tabular}[t]{@{}l}
  {\tt (class #1} #2 {\tt (super-new))}
\end{tabular}
\end{minipage}}

\def\classsupernew#1{%
\begin{minipage}{3.9cm}
%  \begin{tabular}[t]{@{}l} {\tt (class object\% #1)} \end{tabular}
{\tt (class a\% #1)}
\end{minipage}}

\begin{figure*} \footnotesize
  \begin{tabular}{@{}p{2.1cm}@{\,\,}|@{\,\,}p{5.01cm}@{\,\,}|@{\,\,}p{4.99cm}@{\,\,}|@{\,\,}p{1.1cm}@{} }
    {\bf name} & {\bf description} & {\bf example} & {\bf origin}\\ \hline
%% -----------------------------------------------------------------------------
{\tt constant}
 & swaps a constant with another of the same value but different type
 & \example{5.6} -> {5.6+0.0i} 
 & \originspecial 

{\tt deletion}
 & deletes the final expression from a sequence
 & \example{{\tt (begin x y z)}\\} -> {{\tt (begin x y)}}
 & \originspecial 

{\tt position}
  & swaps two sub-expressions
  & \example{{\tt (f a 42 "b" 0)}\\} -> {{\tt (f a 42 0 "b")}}
  & \origingen 

{\tt list}
 & replaces {\tt append} with {\tt cons}
 & \example{{\tt append}} -> {{\tt cons}} 
 & \originnew 
{\tt top-level-id}
 & swaps identifiers defined in the same module
 & \example{{\tt (f x 42)}} -> {{\tt (g x 42)}} 
 & \originnew  

{\tt imported-id}
 & swaps identifiers imported from the same module
 & \example{{\tt (f x 42)}} -> {{\tt (g x 42)}} 
 & \originnew   

{\tt method-id}
 & swaps two method identifiers
 & \example{{\tt (send o f x 42)}\\} -> {{\tt (send o g x 42)}} 
 & \originnew   


{\tt field-id}
 & swaps two field identifiers
 & \example{{\tt (get-field o f)}\\} -> {{\tt (get-field o g)}} 
 & \originnew   
  

{\tt class:init}
 & swaps values of class initializers
 & \example{{\tt (new c [a 5] [b "hello"])}\\} -> {{\tt (new c [a "hello"] [b 5])}} 
 & \originnew   

{\tt class:parent}
 & replaces the parent of classes with {\tt object\%}
 & \example{\classsuperclass{a\%}{}} -> {\\ \classsuperclass{object\%}{\\ \quad}}
 &  \originnew    
 

{\tt class:public}
 & makes a public method private and vice versa
 & \example{\classpr{public}\\} -> {\classpr{private}}
 & \origingen  

{\tt class:super}
 & removes {\tt super-new} calls.
 & \example{\classsupernew{(super-new)}\\} -> {\classsupernew{(void)}} 
 & \originnew

{\tt arithmetic}
 & swaps arithmetic operators
 & \example{{\tt +}} -> {{\tt -}}
 & \origingen

{\tt boolean}
 & swaps {\tt and} and {\tt or}
 & \example{{\tt and}} -> {{\tt or}}
 & \originprevious

{\tt negate-cond}
 & negates conditional test expressions
 & \example{{\tt (if (= x 0) t e)}\\} -> {{\tt (if (not (= x 0)) t e)}}
 & \originprevious

{\tt force-cond}
 & replaces conditional test expressions with {\tt \#t}
 & \example{{\tt (if (= x 0) t e)} \\} -> {{\tt (if \#t t e)}}
 & \originnew

\end{tabular}\\[1ex]

\explanation

\caption{Summary of mutators} \label{table:mutation-ops}
\end{figure*}

\sub{interesting}{Are These Mutators Interesting} 
