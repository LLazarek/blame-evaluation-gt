The starting point for our corpus of programs
is~\citet{gtnffvf-jfp-2019}'s gradual typing benchmark suite. The
benchmark suite consists of fully typed correct programs that are written by different authors
and have been used, maintained and evolved by their authors and others over a
number of years.  They range widely in the size, complexity, purpose and
features of Typed Racket they employ.  Furthermore the benchmarks use advanced aspects
of Typed Racket's type system such as occurrence
typing~\cite{tf-icfp-2010}, types for mutable and immutable data
structures, polymorphic types, types for first-class classes and objects, and types for
Racket's numeric tower~\cite{stathff-padl-12}. Without loss of the diversity of the benchmarks,
we select the ten largest in terms of numbers of components (between 6 and 14 components).
These benchmarks also have the most complex dependency graphs and, hence,
can make the debugging process the hardest for the rational programmer.
The table in figure~\ref{table:benchmark-descriptions} shows our selection
together with a short description for each benchmark.

\begin{figure}
\begin{tabular}{p{1.75cm} | p{6cm} }
  {\bf  name} & {\bf description (author)}  \\

\hline
  \texttt{acquire} & Object-oriented board game implementation (M. Felleisen)  \\%[1em]

\hline
  \texttt{gregor} & Utilities for calendar dates (J. Zeppieri) \\%[1em]

\hline
  \texttt{kcfa} & Functional implementation of 2CFA for the lambda calculus (M. Might) \\%[1em]

\hline
  \texttt{quadT} & Converter from S-expression source code to PDF format (M. Butterick)\\%[1em]

\hline
  \texttt{quadU} & Converter from S-expression source code to PDF format  (B. Greenman) \\%[1em]

\hline
  \texttt{snake} & Functional implementation of the  Snake video game (D. Van Horn) \\%[1em]

\hline
  \texttt{suffixtree} & Algorithm for common longest subsequences between strings. (D. Yoo) \\%[1em]

\hline
  \texttt{synth} & Converter of notes and drum beats to WAV (V. St-Amour \& N. Toronto) \\%[1em]

\hline
  \texttt{take5} & Mixin-based card game simulator (M.Felleisen)  \\%[1em]

\hline
  \texttt{tetris} & Functional implementation of Tetris (D. Van Horn) \\%[1em]

\end{tabular}
  \caption{Benchmarks summary.}
  \label{table:benchmark-descriptions}
\end{figure}

