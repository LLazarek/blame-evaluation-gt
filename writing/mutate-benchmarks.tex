%% -----------------------------------------------------------------------------

Since this is the first empirical attempt to evaluate blame in gradual typing,
there is no standard benchmark suite that we can use off-the-self.  In addition,
there is no exsitng benchmark suite of gradually typed programs with type-level
mistakes.\footnote{\citet{cc-oopsla-20} created a collection of Reticulated
Python programs to evaluate their technique of fixing mistakes in type
annotations. Their collection does not come with type-level mistakes in the code
itself.} In fact, the bar for testing the rational programmer is much higher
than the existence of a collection of programs with type-level errors. Testing
the rational programmer demands a currated collection of programs that take
advantage of the sophisticated features of a gradually typed language such as
Typed Racket.

Over the past decade,  \citet{gtnffvf-jfp-2019} have curated such a
collection of Typed Racket programs for systematically measuring the
implementation's performance. The benchmark suite consists of fully typed,
correct programs, written by a number of different authors, who have
maintained and evolved these programs over several years. The programs
range widely in size, complexity, purpose, origin and in programming
style. The latter is mostly a reference to their reliance on many Typed
Racket features: occurrence typing~\cite{tf-icfp-2010}, types for mutable
and immutable data structures~\cite{hpst-sfp-2010}, types for first-class
classes and objects~\cite{tsdtf-oopsla-2012}, and types for Racket's
numeric tower~\cite{stathff-padl-12}. Hence they offer a diverse lignustic
substrate suitable for injecting type-level errors and testing the
rational programmer. 

Since mixing of typed and untyped code in Typed Racket takes place at the
module level, the experiment calls for benchmarks with a decent number of
modules and a variety of module dependency graphs. Filtering the benchmark
suite using these criteria while preserving its linguistic diversity
yields ten suitable programs. Figure~\ref{table:benchmark-descriptions}
displays the selected benchmark programs and describes their basic
characteristic. 
