%% -----------------------------------------------------------------------------

While at first glance the method of section~\ref{sec:why-rational} seems
straightforward to implement and test, three challenges remain. The first is
comparing the effect of blame on the rational programmer across three different
mechanisms; second is the collection of a large number of representative
debugging scenarios; and third is finding a proper sampling strategy for the
huge space of possibilities. A coincidental challenge is the disparity of the
implementations of gradually typed languages. To eliminate this variable, the
authors use Racket, which is thus far the only language in which all three major
soundness mechanisms are available in a robust and comparable manner:
Typed Racket implements Natural, Shallow Racket~\cite{ttt21}
Transient, and plain Racket for Erasure. 

The {\em first challenge\/} stems from the differences between the blame
assignment mechanisms of the three semantic variants.  While Natural assigns
blame to {\em one\/} component, Transient assigns blame to a sequence of
components. The Erasure semantics does not blame components {\it per se\/}, but
it comes with an exception location and a stack trace, which implicitly suggests
parties to blame.  Each strategy affects the behavior of the rational programmer
differently (and real ones, too).

One way to reconcile these differences is to {\em equip the rational
programmer with modes \/} that represent the different types of
information the rational programmer takes into account when debugging a
scenario. Intuitively, with the common vocabulary of modes, different
blame strategies become different ways the rational programmer  interprets
blame.  For instance, one Transient mode may pick the oldest element of a
blame sequence as the next component to add types because it corresponds to the
earliest point in the program's evaluation that can discover an impedance
mismatch.  Another mode may opt to treat the sequence as a stack and add
types to its newest element.  If both modes are equally successful in
locating an impedance mismatch, measuring the rational programmer's effort with each mode
may answer which is the most effective.


However, attributing the success of the rational programmer to this or
that blame mechanism demands a careful look at the interplay between blame
and the run-time checks of each gradual type system. When a check fails
the Natural and Transient semantics assign blame instead of using the
information in the exception from the run-time check. Nevertheless, the
exception information may be as useful to the rational
programmer as a blame assignment. If this is the case, then blame may not play a critical
role for the rational programmer. Indeed, precisely because they do not
account for such confounding factors, \citet{lksfd-popl-2020} cannot draw any
conclusions about blame specifically, despite advertisements for the
opposite. Their experiment may conclude only that so-called blame-shifting
works, but they cannot attribute this conclusion to blame. 

Modes offer a uniform way to compare the different semantics and  isolate
blame from the effect of the semantics' run-time checks. Specifically, the Natural and
Transient rational programmer come with a blame mode and an exception
mode. If the former succeeds in debugging a program while the latter
fails, then it is safe to conclude that blame is indeed necessary for the
rational programmer. Put differently, the exception mode serves as the
{\em baseline\/} for blame's value within a given semantics; if the
programmer in this mode performs as well or better than the  blame one, a blame assignment
mechanism might be useless.

An experiment must rule out that the usefulness of blame assignment is sheer
luck.  Hence, a random rational programmer mode provides yet another necessary
baseline using the same ingredients as for exceptions.

The {\em second challenge\/} is to find a representative collection of
programs with impedance mismatches. The latter must represent mistakes that programmers
accidentally create and that the run-time checks of academic systems catch. In
other words, the experiment calls for a collection of type mistakes in
mix-typed programs that is representative of those ``in the wild.''
Unfortunately no such collection exists, and with good reason. The kind of
mistakes needed are typically detected by unit or integration tests; even if it
takes some time to find their sources, these mistakes do not make it into code
repositories with appropriate commit messages.

An alternative is to {\em generate a large and representative corpus of type
mistakes \/} using mutation analysis~\cite{lipton1971fault, demillo1978hints,
jia2011analysis}. Conventional mutation analysis is useless, however.  Mutation
analysis traditionally aims to inject bugs that challenge test suites, and it
discards those that yield ill-typed mutants as \emph{incompetent}. Indeed,
mutation analysis frameworks are fine-tuned to avoid them, and yet, it is
precisely those mutators that are needed for evaluating blame assignment strategies.

Based on a related experience, Gopinath and Walkingshaw~\cite{gw-mutation} write,
``existing mutation frameworks \ldots\ do not generate the kinds of mutations
needed to best evaluate type annotations'' and, worse, ``it is surprisingly
difficult to come up with mutants that actually describe subtle type faults.''
While the goal of their work---to evaluate the quality of types in
Python---is unrelated to the one here, the mechanism is related. And their
judgment confirms the experience of the authors. 

An experimental analysis of blame needs a mostly new set of mutators.
Roughly speaking, the new mutators inject type errors into fully typed programs.
Applying such a mutator to any typed component produces a mutated component.  A
debugging scenario results from removing the types from the mutated
component. For the design of such mutators, the authors relied on their own
extensive programming experience though not without discovering a major pitfall:
some of their original mutators systematically produced programs that
immediately revealed the source of the impedance mismatch.  All of the remaining
ones yield {\em interesting debugging scenarios\/} (see
sec.~\ref{sub:mutate-interesting}).

The {\em third challenge\/} is the explosive number of debugging scenarios that
result from the combination of mutation-based scenario generation and mode-based
analysis. All three factors---three different gradual typing systems, the large
number of mutants, and the number of debugging modes---contribute possibly
useful experimental data in a multiplicative manner. Hence, carried out naively,
the experiment would demand an infeasible amount of computational
resources.  A practical execution has no option but to {\em sample the space of
scenarios\/}, carefully ensuring that it still has reproducible
conclusions.

The next three sections explain how to overcome the three challenges
in detail. 
