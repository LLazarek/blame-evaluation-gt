%% -----------------------------------------------------------------------------

The paper provides the first insight into the measurable value of blame in the
gradually typed world. In an implicit manner researchers and language designers
have answered this question one way or another, but without evidence. Now they
may use the evaluation method of this paper to rationalize their answers.  The
work of~\citet{tgpk-dls-2018} suggests that, by intuition, programmers prefer
the run time checking of Natural over other soundness methods. But just because
a random set of programmers expresses this view does not mean that blame
assignment adds value. At this point it is clear that in some circumstances
it does, and in others the answer remains murky.

The paper does {\em not\/} address a problem in the gradually typed world that
was pointed out early on by practical researchers~\cite{incorrect-ts,
sta-nt-base-types, wmwz-ecoop-2017} and that has recently received theoretical
attention~\cite{gfd-oopsla-2019, cc-oopsla-20}: mistakes in type annotations
themselves.  Developers use gradual typing to move an untyped code base into the
typed realm, and to this end, they need typed APIs for the vast repositories of
already-existing libraries. Because they can, language implementors merely
create typed facade modules that import untyped functions and export them with
type annotations---instead of converting the libraries themselves. With those
facades, the compiler can easily type-check typed components. The problem is
that these retroactive additions of types to a library may result from a
misunderstanding of the code. \emph{The ascribed types may thus be a mistake
themselves.}

The cited evidence suggests that this scenario is quite common and largely
unadressed.  A future evaluation must develop mutators that produce incorrect
type annotations without breaking the code itself. Some preliminary work
suggests that such type mutators are even more difficult to develop than the
type-level code mutators presented here. Based on the gradual typing literature,
the Natural semantics should usually discover such mistakes. In contrast, the
Transient and Erasure semantics cannot help with such mistakes at all; indeed,
we expect that these latter two raise misleading exceptions or produce
plainly-wrong answers.  Of course, only further evaluation work can confirm or
reject these conjectures.



