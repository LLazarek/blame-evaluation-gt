%% -----------------------------------------------------------------------------
The paper provides the first insight into the measurable value of blame
 in the gradually typed world. In an implicit manner researchers and
language designers have answered this question one way or another, without any
evidence. Now they may use the evaluation method of this paper to
confirm or reject their answers. We already know from~\citet{tgpk-dls-2018}
that, by intuition, programmers prefer the run-time checking of Natural over
other soundness methods. But, just because a random set of programmers expresses
this view does not mean that blame assignment adds value. At this point we know
that it does, in some circumstances, and that in others, the answer remains
murky.

The paper does {\em not\/} address a problem in the gradually typed world that
was pointed out early on by practical researchers~\cite{incorrect-ts,
sta-nt-base-types, wmwz-ecoop-2017} and that has recently received theoretical
attention~\cite{gfd-oopsla-2019, cc-oopsla-20}: mistakes in type annotations themselves.
Developers use gradual typing to move an untyped code base into the typed realm,
and to this end, they need typed APIs for the vast repositories of
already-existing libraries. Because they can, language implementors merely
create typed facade modules that import untyped functions and export them with
type annotations---instead of converting the libraries themselves. With those
facades, the compiler can easily type-check typed components. The problem is
that these retroactive additions of types to a library may result from a misunderstanding of the
code. \emph{They may thus be a mistake themselves.}

The cited evidence suggests that this scenario is quite common and largely unadressed.
%, and that a type soundness check does {\em not\/} at all address the problem.  
Hence, a future evaluation must develop type-level mutators that produce incorrect type
annotations without breaking the code itself. Some preliminary work suggests
that such type mutators are much more difficult to develop than the code
mutators of this paper. We conjecture that the Natural semantics will
usually discover such mistakes, while the Transient and Erasure semantics cannot
help with such mistakes at all; indeed, we expect that these latter two raise
misleading exceptions or produce plainly-wrong answers.  Of course, only further
evaluation work can confirm or reject these conjectures.



