%% -----------------------------------------------------------------------------

The three blame strategies rely on three different ways of catching problems
with types at run-time: {\em Natural\/}, {\em Transient\/}, and {\em
Erasure\/}. This self-contained section summarizes these options with one
illustrative example using (Typed) Racket syntax. The informed reader may wish to merely
scan it.

Consider the program sketch in figure~\ref{fig:example}. Each box represents a
module: the top bar lists the name and whether it is using typed (\typecolor) or
untyped (\dyncolor) syntax.
\begin{description}

\item[\texttt{pack-lib}] (at the top right) represents a library that provides,
among others, a function {\tt pack}. The documentation says this function
consumes JSON data and packages it in an association list. In reality, though, the
function returns a hash table instead of the association list. 

\item[\texttt{types}] (at the top left) is one of three modules that overlays
types onto this library. This specific modules defines types in common to the
two other typed libraries. 

\item[\texttt{typed-pack-lib}] (at the mid-level on the left) imports {\tt pack}
and re-exports it as \texttt{typed-pack} asserting that it is a function that
consumes {\tt JSON} and returns a list associating {\tt Symbol}s with {\tt
String}s. In other words, it formalizes the comments in {\tt pack-lib}.

\item[\texttt{crypto-pack-lib}] (at the bottom left) also imports \texttt{pack}
and assumes for it the same type as {\tt typed-pack-lib}. It applies the
function in the definition of the exported {\tt crypto-pack} function, which
encrypts its input before passing it to \texttt{pack}.

\item[\texttt{client}] (at the bottom right) uses {\tt pack}
indirectly. Specifically, it goes through the two intermediary typed modules to
use it. Imagine a programmer who relies on the types in the \typecolor\
modules as checked documentation but prototypes the client in the untyped language.

\end{description}

The mistaken comment in {\tt pack-lib} causes an impedance mismatch, with which each
of the three semantics deals differently.  Under {\it the Natural semantics\/},
functions imported into and exported from typed modules are wrapped in proxies that
enforce the static type discipline with run-time checks and track
responsibilities~\citep{tf-popl-2008, tfffgksst-snapl-2017}. Thus, when {\tt pack}
is imported into a typed module, the run-time system checks that it is a function
and wraps it in a protective proxy, which in turn, enforces the type of the function
result with run-time checks.  Analogously the run-time system wraps each exported
function of a typed module such as {\tt crypto-pack} in a proxy that checks its
arguments.  These checks protect functions exported from typed modules against
applications to wrong arguments in untyped code.

As this analysis implies, if a return-type check fails, the problem is
that the untyped module, here {\tt pack-lib}, supplied a function that is not a
match for the type ascribed by the typed module.
Hence either the type at the boundary between the two modules is wrong or, 
if the programmer trusts the type, the untyped module is at fault.
If an argument-type check fails, responsibility  lies with
{\tt client}. After all, either the type it ascribes to the argument is
wrong or the argument it produced clashes with the type. Due to 
proxies, {\it Natural\/} can easily track the 
boundary, type, and responsible parties that correspond to each check. 
Thus, in the example of figure~\ref{fig:example}, as
\texttt{pack} returns, the return-type check fails and Natural blames
the boundary between \texttt{pack-lib} with {\tt typed-pack-lib} and
{\tt crypto-pack-lib}, respectively, for the two {\tt define}s in {\tt client}.

Under {\it Transient\/}, typed code is compiled so that all entry points to
functions check their arguments at run time and all function calls check their
return values against the expected type~\citep{vss-popl-2017}.  Furthermore,
Transient uses \emph{shallow} checks, meaning they inspect only the top
constructor of a value. Since retrieving a value from within a structure (or
list, array, hash table etc.) is performed via a function call, the contents of
a complex value is checked on a piecemeal basis.

As a result, the call to \texttt{typed-pack} does {\em not\/} signal an error
because it takes place in the untyped {\tt client} module, which is compiled in
the usual manner. Because {\tt pack} is called in the {\tt crypto-pack-lib}
module, Transient's inlined checks make sure that the imported
\texttt{pack} is a function and that its result is a list. This
last check fails in \texttt{client}'s call to {\tt crypto-pack}.

In order to locate the corresponding boundaries for failed checks, Transient
maintains a map from values to the boundaries between typed and untyped modules that
they cross, plus the corresponding types. In the example, the map records that
\texttt{pack} crosses from \texttt{pack-lib} to {\tt typed-pack-lib} and
from {\tt pack-lib} to {\tt crypto-pack-lib} with the type that appears in the
{\tt required/typed} forms in the example. Since the failed check corresponds to
the return type of \texttt{pack}, assuming that the type is correct,
the responsible party is the source of the two
boundary crossings: \texttt{pack-lib}. In
general though, Transient blames more than one boundary. In fact, the theoretical
work of~\citet{gfd-oopsla-2019} shows that for some programs Transient
constructs a blame sequence that excludes responsible parties and includes modules
irrelevant to the failing check.

Under {\it Erasure\/}, the compiler checks the specified types and
then discards them when it generates code. The generated code includes run-time
checks that ensure the dynamic safety of all operations as specified in the
underlying untyped language. Hence, neither the call to {\tt typed-pack} nor the
call to {\tt crypto-pack} signals an error due to the gradual type system. If
at some later point {\tt client} tries to inspect the elements of the lists
that \texttt{typed-pack} and {\tt crypto-pack} are supposed to produce, Racket's
safety checks signal a violation and point to some place in {\tt client}. The
information in this exception, plus its stack trace, may help the programmer
find the source of the impedance mismatch between the specified types of
{\tt pack} in the two typed modules and its actual results.

The following table summarizes the illustration. Each cell describes the result
of evaluating the column's definition (in {\tt client}) under the row's
semantics.
%% -----------------------------------------------------------------------------
\begin{center}
  \begin{tabular}{l|l@{\qquad}l}
                  & {\tt public-data}              & {\tt secret-data}                                 \\
\hline %% ----------------------------------------------------------------------------------------
{\it Natural\/}   & error, blaming the boundary between & error, blaming the boundary between          \\
                  & {\tt pack-lib} and {\tt typed-pack-lib} & {\tt pack-lib} and {\tt crypto-pack-lib} \\
\hline
{\it Transient\/} & no error                       & error, blaming the boundaries between            \\
                  &                                & {\tt pack-lib}  and {\tt typed-pack-lib}/ \\
                  &                                & {\tt  pack-lib} and {\tt crypto-pack-lib}     \\
\hline
 {\it Erasure\/}   & no error*                      & no error*                                         \\
                  &   \multicolumn{2}{|c}{*but {\em Erasure} does signal an error on list access}
\end{tabular}
\end{center}
%% -----------------------------------------------------------------------------


