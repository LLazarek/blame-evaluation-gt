\section{How to Bring Natural, Transient and Erasure Under the Same Roof} \label{sec:landscape}


   By example; adjusted version of the one in OOPSLA sec 2

  \begin{itemize}

  \item Why Blame Assignments Differ.

  \item How Blame Assignments Differ.

  \item A space of GT system design. In terms of error reporting, we identify two axes of GT system
        design: where checking occurs and whether blame is tracked.
    

  \end{itemize}



\subsection{Natural Typed Racket}

Typed Racket extends the untyped Racket language with \emph{natural}\/ gradual
 types~\cite{}.
Natural types guarantee a strict separation between typed and untyped code;
 informally, an untyped value can interact with a typed context only if the
 value behaves in a well-typed way.
Typed Racket implements this separation by compiling types to higher-order
 contracts that guard all boundaries to untyped code.
Any higher-order typed value that exits a boundary receives a protective
 wrapper, and any untyped value that enters must pass a deep run-time type
 check.
These contract checks help predict the behavior of typed code, but
 may greatly slow down a program.


\subsection{Transient Typed Racket}

Transient gradual types ensure that typed code cannot get stuck on a primitive
 operation.
This assurance can be implemented without higher-order
 contracts~\cite{vss-popl-2017}, and therefore offers a modicum of type
 soundness for a seemingly-low implementation and performance cost.

For this experiment, we modified Typed Racket to bring transient and natural
 under the same roof.
Our version of the library is parameterized by a \emph{type-enforcement strategy}
 to allow reuse of common components.
Under the \emph{natural} strategy, our Typed Racket is identical to the latest release.
By contrast, the \emph{transient} strategy re-uses the static type checker and
 much of the compiler, but weakens the run-time meaning of types in two ways.

The first major ingredient in transient Typed Racket is a new interpretation
 of types as first-order shape checks.
Each shape-check enforces a type constructor.
For certain types, such as {\tt String} or {\tt (Vectorof Byte)}, transient can
 re-use a predicate from the underlying Racket language.
Other types require custom predicates; for example the type {\tt Index} of
 valid array indices allows a range of positive exact integers.
Many custom predicates run in constant time, but some are more involved.
One linear-time check is for {\tt (Listof T)} for any element type {\tt T},
 because it must distinguish a null-terminated list of pairs from improper
 list-like structures.
The key point, however, is that no transient check relies on
 higher-order contract wrappers.

Second, transient adds a new pass to the compiler that rewrites all typed code.
Rewriting changes all elimination forms to shape-check the result.
For example, if the variable {\tt f} has the static type {\tt (-> String String)}
 then the application {\tt (f x}} must be rewritten to check for a string result
 because {\tt f} could be an untyped function.
Rewriting also inserts hooks to update a global blame heap.
When a value crosses a boundary, the blame heap records a cast assumption.
Elimination forms and other operations record smaller path assumptions.
If a transient check ever fails, the language follows path assumptions and
 reports a set of casts/boundaries to the programmer.



