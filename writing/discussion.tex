%% -----------------------------------------------------------------------------
%% \begin{itemize}
%% \item Why does the data look like this?
%%   - why tr fails
%%     - bad runtime errors
%%     - TR bug
%%   - why transient fails
%%     - bad runtime errors
%%     - timeouts/ooms: give context wrt fully untyped/typed versions
%%     - empty blame: trim Ben's prose

%%   \item Any kind of dynamic checking has a large positive impact, whether blame or stacktraces are used.
%%     \begin{itemize}
%%     \item Both in terms of usefulness and programmer effort
%%     \item Erasure extremely often fails to provide any helpful information.
%%     \item That said, there are no long trails for Erasure.
%%     \end{itemize}

%%   \item Blame helps significantly over just using stacktraces for both Natural and Transient.
%%     \begin{itemize}
%%     \item Natural's blame improves over exceptions by nearly 20\%
%%     \item Transient's blame improves over exceptions half as much
%%     \end{itemize}

%%   \item Natural clearly offers the most utility, but Transient is not too far behind.
%%     \begin{itemize}
%%     \item Natural clearly improves over all other modes
%%     \item Transient improves significantly over all other modes except Natural.
%%     \item Keep in mind that our selection of scenarios is biased toward Erasure & Transient.
%%     \end{itemize}
    

%%   \item Discussion of how confidence / margin of error should be interpreted wrt these results?


%%    \item Threats to validity:
%%         \begin{itemize}
%%           \item Racket stack traces
%%           \item Mutant sampling
%%           \item Blame trails do not necessarily capture all aspects of error reporting
%%           \item Transient blame adaptations
%%           \item More?

%%    \end{itemize}
%% \end{itemize}       



The analysis in the previous section suggests a number of interesting
high-level conclusions about blame in the gradual typing setting.  First,
run time type checks have a large positive impact. This is the case
regardless of whether these checks issue blame or throw plain
exceptions.  Second, in many cases type checks that issue blame are more
helpful than those that do not. However, the results indicate that blame
is not critical in a majority of cases, and therefore it is worth further
investigation whether disabling blame tracking for the sake of performance
may be   an acceptable trade-off. Third, from the three different blame
strategies, the Natural approach of Typed Racket fares better than the
Transient approach of Reticulated Python, but only by a small margin.
Since Natural offers complete and sound path-based blame while Transient
offers incomplete but sound heap-based blame~\cite{gfd-oopsla-2019}, the
results call for a deeper understanding of the relative strengths of the
two models for blame.  Fourth, given that Transient's sound but shallow
run time type checks do not seem to hamper debugging, a version of Typed
Racket that disables some of its wrappers may offer an answer to the
well-known performance issues of Natural~\cite{gtnffvf-jfp-2019}.  Fifth,
just as for Typed Racket, the experiment points to possible improvements
for Reticulated Python. In particular, the experimental data indicates
that the usefulness of transient blame does not depend on whether  the
rational programmer follows first or the last element from a blame set,
and that using the last element seems to produce shorter trails.  Hence,
given the performance issues of Transient discussed further on, a possible
optimization is to limit the size of blame sets to only contain the latest
entries using timestamps.

There are a number of threats to the validity of these conclusions: (i) the
representativeness of benchmarks; (ii) the relation between mutations and
real programming mistakes; (iii) the definition of interesting debugging
scenarios; and (iv) the sampling strategy. The previous sections
have described the various ways we attempt to mitigate these threats;
the remainder of this section discussed three additional threats.

%% -----------------------------------------------------------------------------
\subsection{Threat: Is the Rational Programmer Realistic?}

Like {\em homo economicus\/}, which decouples the actual behavior of a
participant in an economy for the sake of mathematical modeling, the model of a
rational programmer decouples the actual debugging behavior of a software
developer for the sake of a systematic, large-scale analysis. This decoupling
comes with advantages and disadvantages. In the economic realm, mathematical models
have provided some predictive insights into the market's behavior; but as
behavioral economics has shown more recently, the mathematical abstraction of a
rational actor makes predictions also quite unreliable in some situations.
\footnote{It has also misled economists to focus on just the mathematics, though
this problem is not relevant here.}  Just like an ordinary consumer or producer,
an actual software developer is unlikely to stick to the exact strategy proposed
here. When this happens, the predicted benefits of blame assignment may not
materialize. Indeed, the authors's personal experience suggests such deviations, and
it also suggests that deviating is a mistake. To make a true judgment of the
usefulness of the rational-programmer idea, the community will need a lot more
experience with this form of evaluation and relating the evaluation to the
behavior of working programmers.

Relatedly, the experimental setup hides the type choice a rational programmer must make. When the
run-time checks signal an impedance mismatch in the real world, a programmer
does not have a typed component ready to swap in. Instead, the programmer must
come up with the next set of types---and this means the programmer must make
choices. It is normally possible to assign consistent types to variables in a
component in different ways. The creation and curation of the benchmarks over
many years has driven home this lesson, but fortunately, it has also shown that
the types are in most cases reasonably canonical.  The authors therefore conjecture that
a rational programmer would in most cases come up with an equivalent type
assignment in order to extend a trail as the experimental method describes.

%% -----------------------------------------------------------------------------
\subsection{Threat: Why Does Transient Lose Blame?} \label{sec:threat:transient}

The results in section~\ref{sec:results} show that Transient produces empty
blame sets for 967 scenarios. In theory, empty set must not occur. An
empty blame set means a lack of boundary crossings for the witness value.
By implication, a typed module is blaming itself---something that can
happen only if the type checker (or system) is unsound.

An investigation of these empty blame cases reveals that Transient is not
unsound but that Vitousek et al.'s Transient blame assignment loses track
of the proper boundaries due to an incomplete population of the blame map.
Despite these losses, the number of failures is a small fraction of the
overall number of debugging scenarios. It is unlikely that any
improvements to Vitousek et al.'s strategy would change the overall
conclusion about Transient blame.

{\bf Note} The scenarios where Transient returns empty blame 
suggest ways to improve Transient's blame map as follows:
\begin{itemize}

\item Entries in a blame map should point to {\em several\/} parent entries.
 For example, if \texttt{f} receives bad input in the context of {\tt (filter f
 xs)}, the entry should point to both \texttt{filter} and \texttt{xs}.

\item The construction of entries must be guided by type-like specifications for
 primitives instead of special cases for built-in functions. Aliasing such
 function changes the blame map.

\end{itemize}
This small example is indicative of how rational-programmer experiments may
provide actionable feedback for language design . 


%% -----------------------------------------------------------------------------
\subsection{Threat: Is the Transient Blame Assignment Mechanism Realistic?}
\label{sec:threat:transient2}

The results in section~\ref{sec:results} also show that the cost of Transient
blame is quite high. Under the Transient semantics, some of the debugging
scenarios exceed the 4-minute timeout or the 6GB-memory limit. To put those
limits into context, the fully typed and fully untyped benchmarks all complete
in a few seconds with minimal memory usage. Furthermore, none of the mixed
Natural configurations hit these limits, and with the blame map turned off,
the Transient semantics also runs these programs in a short amount of time and
well within the memory limits. In short, even though the Transient rational
programmer appears to do well in the experiment, the implementation of the
Transient blame strategy might be unrealistic. 

At first glance, these measurements seem to contradict the results
of~\citet{vss-popl-2017}. They report an average slowdown of 2.5x and a
worst-case of 5.4x on fully-typed benchmarks in Reticulated Python.
Unfortunately, the high cost of Transient Racket seems closer to the truth.
There are at least four broad issues that skew the earlier results:
\begin{enumerate}

\item The 2017 implementation of Reticulated fails to insert certain soundness
checks\footnote{Missing check:
\url{https://github.com/mvitousek/reticulated/issues/36}} and blame-map
updates\footnote{Missing cast:
\url{https://github.com/mvitousek/reticulated/issues/43}} from the paper.

\item The types of almost all local variables in the "fully typed" benchmarks is
{\tt Dynamic}, often due the impoverished nature of the type system, which does
not allow the ascription of precise types.  Code that ends up with the type {\tt
Dynamic} has fewer constraints to check at run-time---and much less information
to track in the blame map.

\item \citet{vss-popl-2017} use small benchmarks.  Four have since been retired
from the official Python benchmark suite because they are too small,
unrealistic, and unstable.\footnote{Release notes:
\url{https://pyperformance.readthedocs.io/changelog.html}}

\item On large programs, Reticulated suffers from high overhead.  For example, a
Reticulated version of the simplest benchmark runs in ~40 seconds without blame,
and with blame times out after 10 minutes. Typed Racket runs the same benchmark
in a small fraction of the time.
\end{enumerate}
More work on Transient blame is needed to make an informed decision. 


%% NOTE: Vitousek's benchmarks are from the Python "pyperformance" suite.
%%   The version notes in the docs talk about retiring benchmarks, but
%%   you can also look at the current codebase and see what names from POPL'17
%%   are missing: callsimple, callmethod, callmethodslots, & pystone
%% <https://github.com/python/pyperformance/tree/master/pyperformance/benchmarks>
