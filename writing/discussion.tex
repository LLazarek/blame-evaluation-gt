%% -----------------------------------------------------------------------------
\section{What We Learned}
\label{sec:discussion}

%% \begin{itemize}
%% \item Why does the data look like this?
%%   - why tr fails
%%     - bad runtime errors
%%     - TR bug
%%   - why transient fails
%%     - bad runtime errors
%%     - timeouts/ooms: give context wrt fully untyped/typed versions
%%     - empty blame: trim Ben's prose

%%   \item Any kind of dynamic checking has a large positive impact, whether blame or stacktraces are used.
%%     \begin{itemize}
%%     \item Both in terms of usefulness and programmer effort
%%     \item Erasure extremely often fails to provide any helpful information.
%%     \item That said, there are no long trails for Erasure.
%%     \end{itemize}

%%   \item Blame helps significantly over just using stacktraces for both Natural and Transient.
%%     \begin{itemize}
%%     \item Natural's blame improves over exceptions by nearly 20\%
%%     \item Transient's blame improves over exceptions half as much
%%     \end{itemize}

%%   \item Natural clearly offers the most utility, but Transient is not too far behind.
%%     \begin{itemize}
%%     \item Natural clearly improves over all other modes
%%     \item Transient improves significantly over all other modes except Natural.
%%     \item Keep in mind that our selection of scenarios is biased toward Erasure & Transient.
%%     \end{itemize}
    

%%   \item Discussion of how confidence / margin of error should be interpreted wrt these results?


%%    \item Threats to validity:
%%         \begin{itemize}
%%           \item Racket stack traces
%%           \item Mutant sampling
%%           \item Blame trails do not necessarily capture all aspects of error reporting
%%           \item Transient blame adaptations
%%           \item More?

%%    \end{itemize}
%% \end{itemize}       



The analysis in the previous section suggests a number of interesting
high-level conclusions about blame in the gradual typing setting.  First,
run time type checks have a large positive impact. This is the case
regardless of whether these checks issue blame or throw plain exceptions.
Second, in many cases type checks that issue blame are more helpful than
those that do not. However, our results indicate that in other cases blame
is not critical and thus it is worth further investigation whether
disabling blame tracking for the sake of performance may be   an
acceptable trade off. Third, from the three different blame strategies, the
Natural approach of Typed Racket fares better than the Transient approach
of Reticulated Python, but only by a small margin. Since Natural offers complete
and sound path-based blame while Transient offers incomplete but sound heap-based
blame~\cite{gfd-oopsla-2019}, our results call for a deeper understanding
of the relative strengths of the two models for blame.  Fourth, given that
Transient's sound but shallow run time type checks do not seem to hamper
debugging, a version of Typed Racket that disables some of its wrappers
may offer an answer to the performance issues of Natural. 
Fifth, just as for Typed Racket, the experiment points to possible
improvements for Reticulated Python. In particular, the experimental 
data indicates that the usefulness of transient blame does not depend on
whether  the rational programmer follows first or the last element from a
blame set. Hence, given the performance issues of Transient we discuss further on, 
a possible optimization is to limit the size of blame sets
using timestamps.

There are a number of threats to the validity of our conclusions: (i) the
representativeness of benchmarks; (ii) the relation between mutations and
real programming mistakes; (iii) the definition of interesting debugging
scenarios; and (iv) our approach to sampling scenarios. The previous sections
have described the various ways we attempt to mitigate these threats;
we discuss two additional threats in the remainder of this section.


\subsection{Threat: Is the Rational Programmer Realistic?}

Like {\em homo economicus\/}, which decouples the actual behavior of a
participant in an economy for the sake of mathematical modeling, the model of a
rational programmer decouples the actual debugging behavior of a software
developer for the sake of a systematic, large-scale analysis. This decoupling
comes with advantages and disadvantages. In the economic realm, mathematical models
have provided some predictive insights into the market's behavior; but as
behavioral economics has shown more recently, the mathematical abstraction of a
rational actor makes predictions also quite unreliable in some situations.
\footnote{It has also misled economists to focus on just the mathematics, though
this problem is not relevant here.}  Just like an ordinary consumer or producer,
an actual software developer is unlikely to stick to the exact strategy proposed
here. When this happens, the predicted benefits of blame assignment may not
materialize. Indeed, our own personal experience suggests such deviations, and
it also suggests that deviating is a mistake. To make a true judgment of the
usefulness of the rational-programmer idea, the community will need a lot more
experience with this form of evaluation and relating the evaluation to the
behavior of working programmers.

Relatedly, our setup hides the type choice a rational programmer must make. When the
run-time checks signal an impedance mismatch in the real world, a programmer
does not have a typed component ready to swap in. Instead, the programmer must
come up with the next set of types---and this means the programmer must make
choices. It is normally possible to assign consistent types to variables in a
component in different ways. The creation and curation of the benchmarks over
many years has driven home this lesson, but fortunately, it has also shown that
the types are in most cases reasonably canonical.  We therefore conjecture that
a rational programmer would in most cases come up with an equivalent type
assignment for trail extension as our experimental setup describes.


\subsection{Threat: Is our Transient Correct?}
\label{sec:threat:transient}

As we discuss in section~\ref{sec:results}, Transient surprisingly
produces
an empty blame set in 967 scenarios.
Such cases should never occur, at least in theory, because an empty set
 means the value has never crossed a boundary---if the value is indeed defined
 in the current module, then we have typed code blaming itself for a typed
 value as all the checks are inlined in typed code. Thus, the type checker must be unsound.
After careful investigation of these empty blame cases, we found no soundness
 bugs in Transient Typed Racket.
Instead, we found scenarios in which Transient lost track of the proper
 boundaries.
These scenarios suggest ways to improve~\citet{vss-popl-2017}'s blame
algorithm:
\begin{itemize}
  \item
    Entries in a blame map must point to several parent entries.
    For example, if the function \texttt{f} receives bad input in the call
    \texttt{(filter f xs)}, then blame should point to both \texttt{filter}
    and the \texttt{xs} list.
  \item
    The construction of blame map entries must be guided by type-like specifications
     instead of relying on syntax.
    Aliasing the built-in \texttt{filter} function should not change the shape
     of the blame map.
  \item
    The initial blame map must reflect the initial type environment.
    For instance, Typed Racket trusts that untyped core-library functions, such as \texttt{filter},
     behave correctly; such assumptions should appear in the blame map.
\end{itemize}
\noindent{}Despite these known limitations, based on the relatively small
number of debugging failures, we conjecture that these improvements
 to Transient blame-tracking do not affect our overall conclusions.

A second apparent flaw in our Transient implementation is the high cost of blame.
Whereas \citet{vss-popl-2017} report an average slowdown of 2.5x and
worst-case of 5.4x on fully-typed benchmarks in Reticulated,
some of our configurations exceed our 4 minute timeout or 6GB memory limit with transient blame.
To put those limits into context, the fully typed and fully untyped benchmarks all complete in a few seconds with minimal memory usage, and none of the mixed Natural configurations hit these limits.
Unfortunately, our high cost for Transient seems closer to the truth;
 there are at least three broad issues that skew the earlier results.
First, the 2017 implementation of Reticulated fails to insert certain
 soundness checks\footnote{Missing check: \url{https://github.com/mvitousek/reticulated/issues/36}}
 and blame-map updates\footnote{Missing cast: \url{https://github.com/mvitousek/reticulated/issues/43}}
 from the paper.
Second, Reticulated's type checker assigns the dynamic type to local
 variables in the "fully typed" benchmarks---often because the type system
 cannot articulate a more precise type.
Code that ends up with the dynamic type has fewer constraints to check at run-time.
Third, \citet{vss-popl-2017} use relatively small benchmarks.
Four have since been retired from the official Python benchmark suite
 because they are too small, unrealistic, and unstable.\footnote{Release notes: \url{https://pyperformance.readthedocs.io/changelog.html}}
On larger programs, Reticulated suffers from high overhead.
For example, a Reticulated version of our simplest benchmark runs in
 ~40 seconds without blame, and with blame times out after 10 minutes.

%% NOTE: Vitousek's benchmarks are from the Python "pyperformance" suite.
%%   The version notes in the docs talk about retiring benchmarks, but
%%   you can also look at the current codebase and see what names from POPL'17
%%   are missing: callsimple, callmethod, callmethodslots, & pystone
%% <https://github.com/python/pyperformance/tree/master/pyperformance/benchmarks>



