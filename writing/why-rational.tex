%% -----------------------------------------------------------------------------

The general challenge of evaluating blame is a methodological one. Unlike most
current research on programming languages, the question seems to call for
empirical studies similar to those of the human-computer interaction research
area. At the same time, a significant result demands a large amount of
data. As~\citet{lksfd-popl-2020} recently demonstrated, the way around this
dilemma is to simulate a programmer algorithmically on a large set
of programming scenarios.

%% -----------------------------------------------------------------------------
\begin{figure}\footnotesize

\centerline{\tt (define-type NPR Nonpositive-Real)}

\vspace\spacr

%% ------------------------------------------------------------------
\begin{module}{server : racket}{\dyncolor}
(provide neg-abs)

(define (neg-abs x) (- x)))
\end{module}\hfill\begin{module}{server-typed : typed/racket}{\typecolor}
(provide neg-abs)

(: neg-abs (Real -> NPR))
(define (neg-abs x) (- x))
\end{module}

\vspace\spacr

%% ------------------------------------------------------------------
\begin{module}{layer : racket}{\dyncolor}
(provide na-client)

(require (submod ".." server))

(define (na-client x)
  (* 4 (neg-abs x))))
\end{module}\hfill\begin{module}{layer-typed : typed/racket}{\typecolor}
(provide na-client)

(require/typed
 (submod ".." server)
 [neg-abs (-> Real NPR)])

(: na-client (-> Real NPR))
(define (na-client x)
  (* 4 (neg-abs x)))
\end{module}

\vspace\spacr

%% ------------------------------------------------------------------
\begin{center}\begin{module}{main : typed/racket}{\typecolor}
(require/typed
  (submod ".." layer)
  [na-client (-> Real NPR)])

(define x (na-client -10))
(displayln x)
\end{module}\end{center}

\caption{A simplistic debugging scenario} \label{fig:rational}
\end{figure}
%% -----------------------------------------------------------------------------


This paper generalizes Lazarek et. al.'s idea to {\em the rational
programmer\/}.  Like~\citet{mill1874essays}'s {\it homo economicus\/}, the
rational programmer approximates the behavior of a software developer who
reduces time spent on a task by exploiting the available information. In the
context of gradual typing, the rational programmer has two pieces of information
when an impedance mismatch signals exceptional behavior: the error message and
the state of the program. Hence, the most rational procedure is to use the
former to improve the latter. Specifically, the rational programmer translates
the Wadler--Findler slogan into a debugging method, searching for the source of
the impedance mismatch by adding type annotations to some of the untyped parts
of the program identified in the error.  If the type checker rejects an
annotation derived from the context, the rational programmer has found the
source of the problem.  Otherwise, the rational programmer concludes that the
just-annotated parts are not the problem and re-runs the program---which must,
by the slogan, blame a different location for the problem. At this point, the
rational programmer can iterate the process.  Measuring this simulated behavior
on a large number of scenarios yields data that is similar to data collected in
a human-facing study.

The idea is best illustrated with an example in Typed Racket's migratory type
system. Imagine a code base with dozens of modules in plain Racket. A developer
who opens a module for maintenance purposes must study the module's design and,
as part of the process, is bound to re-construct the types that went into the
module's creation.  To help future maintainers, the developer should report
these insights as type annotations. Over time, the code base migrates into a mix
of typed and untyped modules. As~\citet{tfffgksst-snapl-2017} report though,
it is equally common that developers add typed modules that depend on
the existing modules in the code base.

Now consider the concrete (and simplistic) example of figure~\ref{fig:rational}.
Initially the code base consists of the two \dyncolor\ modules on the left plus
the \typecolor\ module at the bottom; \dyncolor\ indicates untyped, while
\typecolor\ means typed. When a typed module imports an untyped module, it must
assign types to the imported identifiers for the type checker's sake. Here {\tt
main} specifies that {\tt na-client} consumes a {\tt Real} number and produces a
non-positive one.\footnote{Racket's type system reifies
reasoning about subsets of numbers, not machine-level
representations~\citep{stathff-padl-12}.} A program execution ends in this error:
\begin{verbatim}
    na-client: broke its own contract
      promised: (<=/c 0)
      produced: 40
      in: (-> any/c (<=/c 0))           
      contract from: (interface for na-client)
      blaming: (interface for na-client)
       (assuming the contract is correct)
\end{verbatim}
The referenced contract is the compilation of the type of {\tt na-client}. The
definitive hint is ``{\tt blaming: (interface for na-client)}'' with the caveat
``{\tt (assuming the contract is correct)}.''

Assuming the rational programmer trusts the type of {\tt na-client}, the next
step is to inspect the {\tt layer} module and to equip it with type
annotations. The result is the \typecolor\ module in the middle, and {\tt
main}'s import is now re-directed there by {\tt (submod ".." layer-typed)}. As
predicted by the theory, running the modified program (in the same way as before) yields a different error message:
\begin{verbatim}
    neg-abs: broke its own contract
      promised: (<=/c 0)
      produced: 10
      in: (-> any/c (<=/c 0))
      contract from: (interface for neg-abs)
      blaming: (interface for neg-abs)
       (assuming the contract is correct)
\end{verbatim}
Lastly, the rational programmer assigns types to {\tt server} and re-directs
the import of {\tt layer-typed} to {\tt (submod ".." server-typed)}. Now the
type checker objects to the conjectured type of {\tt neg-abs}, i.e.\ the source
of the impedance mismatch is found. How to fix it is a separate question.

Like {\it homo economicus\/}, the rational programmer is an approximation.
People do not behave in a purely rational manner as economic actors, and they
also do not do so as software developers. The point is not to deny the existence
of ``lucky hunches'' programmers or ``tinkering works'' approaches and so on. It
is also not to claim that equipping entire modules with types represents an
always feasible approach.\footnote{Adding types at the expression level, say, as
in TypeScript should be considered well within bounds.}  {\em But\/}, the
concept of studying the idea of an economically rational actor has produced
benefits to the discipline of political economics, and this paper suggests that
implementing and studying the rational programmer will help language designers.

%% -----------------------------------------------------------------------------

%% -----------------------------------------------------------------------------

Every time the rational programmer succeeds, it is validation for programming
language researchers. It shows how their theorems, slogans, and tools help
programmers. The example of blame assignment mechanisms makes this point
clearly. A blame-assignment mechanism provides information that, according to
programming language research, points toward the source of the problem.

When the rational programmer fails, it questions programming
language research. Specifically, it indicates limited predictive
power of programming language theory with respect to the use of languages in
practice. Indeed, misleading predictions may even suggest flaws in language design.

In this way, the idea of a rational programmer implies an entire methodology for
evaluating the design of programming languages. At this point, this study of
methods supplies many questions whose answers might point to suitable evaluation
methods: 

\begin{enumerate} 

\item Does a language design provide information that can guide a
 rational programmer?

\item Does the underlying theory suggest actions to the rational programmer?

\item Can this guidance be formulated as an algorithm?

\item Does the underlying theory lead to a hypothesis about the effects of
 these actions? 

\item Can this hypothesis be tested with a large-scale automated experiment?

\end{enumerate}

Here is how the answers to these methodological questions lead to an evaluation
method for blame-assignment mechanisms: 

\begin{enumerate}

\item The design of blame-assignment mechanisms explicitly advertises the blame
information as helpful for debugging impedance mismatches.

\item The error messages of blame-assignment mechanisms include suspect
locations at the boundary of typed and untyped code fragments.
The Wadler--Findler slogan suggests that the source of the problem is concealed due to a lack of types, so
adding types to the untyped fragment should lead to the source of the impendance mismatch.

\item The step-by-step construction of paths based on error messages from
gradual-typing checks is clearly amenable to implementation, modulo the
ascription of types to modules.

\item The theory conjectures that blame assignment constrains the search
space that a developer must inspect to find the problem.

\item Based on these insights, the remaining sections detail a large-scale automated experiment.

\end{enumerate}
That said, using the method to conduct data-gathering experiments poses
several challenges. The specific challenges are spelled out in the next section,
and the following three sections explain ways of overcoming them.

