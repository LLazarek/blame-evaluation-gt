%% -----------------------------------------------------------------------------

The general challenge of evaluating blame is a methodological one. Unlike most
current research on programming languages, the question seems to call for
empirical research similar to that of human-computer interaction. At the same
time, a significant evaluation result demands a large amount of
data. As~\citet{lksfd-popl-2020} recently demonstrated, the way around this
dilemma is to postulate the {\em rational programmer\/}---the equivalent of a
{\it homo economicus\/} for programming languages---and to implement it as an algorithm.

In the context of gradual typing, the rational programmer translates the
Wadler-Findler slogan into a debugging method, searching for the source of the
impedance mismatch in an incremental fashion. Measuring this simulated
programmer's behavior on a large number of debugging scenarios yields data that
is similar to data collected in an empirical, human-facing manner.

%% -----------------------------------------------------------------------------
\begin{figure}\footnotesize

\centerline{\tt (define-type NPR Nonpositive-Real)}

\vspace\spacr

%% ------------------------------------------------------------------
\begin{module}{server : racket}{\dyncolor}
(provide neg-abs)

(define (neg-abs x) (- x)))
\end{module}\hfill\begin{module}{server-typed : typed/racket}{\typecolor}
(provide neg-abs)

(: neg-abs (Real -> NPR))
(define (neg-abs x) (- x))
\end{module}

\vspace\spacr

%% ------------------------------------------------------------------
\begin{module}{layer : racket}{\dyncolor}
(provide na-client)

(require (submod ".." server))

(define (na-client x)
  (* 4 (neg-abs x))))
\end{module}\hfill\begin{module}{layer : typed/racket}{\typecolor}
(provide na-client)

(require/typed
 (submod ".." server)
 [neg-abs (-> Real NPR)])

(: na-client (-> Real NPR))
(define (na-client x)
  (* 4 (neg-abs x)))
\end{module}

\vspace\spacr

%% ------------------------------------------------------------------
\begin{center}\begin{module}{main : typed/racket}{\typecolor}
(require/typed
  (submod ".." layer)
  [na-client (-> Real NPR)])

(define x (na-client -10))
(displayln x)
\end{module}\end{center}

\caption{A simplistic debugging scenario} \label{fig:rational}
\end{figure}
%% -----------------------------------------------------------------------------

