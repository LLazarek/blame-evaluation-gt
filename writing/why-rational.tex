%% -----------------------------------------------------------------------------

The general challenge of evaluating blame is a methodological one. Unlike most
current research on programming languages, the question seems to call for
empirical research similar to that of human-computer interaction. At the same
time, a significant evaluation result demands a large amount of
data. As~\citet{lksfd-popl-2020} recently demonstrated, the way around this
dilemma is to simulate a programmer's activities algorithmically on a large set
of programming scenarios.

%% -----------------------------------------------------------------------------
\begin{figure}\footnotesize

\centerline{\tt (define-type NPR Nonpositive-Real)}

\vspace\spacr

%% ------------------------------------------------------------------
\begin{module}{server : racket}{\dyncolor}
(provide neg-abs)

(define (neg-abs x) (- x)))
\end{module}\hfill\begin{module}{server-typed : typed/racket}{\typecolor}
(provide neg-abs)

(: neg-abs (Real -> NPR))
(define (neg-abs x) (- x))
\end{module}

\vspace\spacr

%% ------------------------------------------------------------------
\begin{module}{layer : racket}{\dyncolor}
(provide na-client)

(require (submod ".." server))

(define (na-client x)
  (* 4 (neg-abs x))))
\end{module}\hfill\begin{module}{layer-typed : typed/racket}{\typecolor}
(provide na-client)

(require/typed
 (submod ".." server)
 [neg-abs (-> Real NPR)])

(: na-client (-> Real NPR))
(define (na-client x)
  (* 4 (neg-abs x)))
\end{module}

\vspace\spacr

%% ------------------------------------------------------------------
\begin{center}\begin{module}{main : typed/racket}{\typecolor}
(require/typed
  (submod ".." layer)
  [na-client (-> Real NPR)])

(define x (na-client -10))
(displayln x)
\end{module}\end{center}

\caption{A simplistic debugging scenario} \label{fig:rational}
\end{figure}
%% -----------------------------------------------------------------------------


This paper dubs Lazarek et. al's idea {\em the rational programmer\/}. Like Stuart
Mill's {\it homo economicus\/}, the rational programmer approximates the
behavior of a software developer who reduces time spent on a task given the
available information. In the context of gradual typing, the rational programmer
has two pieces of information when an impedance mismatch signals exceptional
behavior: the error message and the state of the program. Hence, the most
rational procedure is to use the former to improve the latter. Specifically,
a rational programmer translates the Wadler-Findler slogan into a debugging
method, searching for the source of the impedance mismatch by adding type
annotations to the parts of the program identified in the error.
If the type checker rejects an annotation, the rational programmer has found the problem.
Measuring this simulated programmer's behavior on a large number of
debugging scenarios yields data that is similar to data collected in an
empirical, human-facing manner.

The idea is most easily illustrated with an example from the Typed Racket
world. In the context of Typed Racket, researchers speak of migratory
typing. Imagine a code base consisting of dozens of modules in plain Racket.  A
developer who modifies a module in this collection for maintenance purposes must
study the module designs---reify the types that went into its creation---and to
help future maintainers should report these insights as type annotations. Over
time, the code base turns into a mix of typed and untyped modules.

Now consider the concrete (and simplistic) example of figure~\ref{fig:rational}.
Initially the code base consists of the two \dyncolor\ modules on the left plus
the \typecolor\ module at the bottom; \dyncolor\ indicates untyped, while
\typecolor\ means typed. When a typed module imports an untyped module, it must
assign types to the imported identifiers for the type checker's sake. Here {\tt
main} specifies that {\tt na-client} consumes a {\tt Real} number and produces a
non-positive one.\footnote{Typed Racket's type system reflects Racket's subset
reasoning about numbers, not (only) machine-level
representations~\cite{stathff-padl-12}.} A program execution ends in this error:
\begin{verbatim}
    na-client: broke its own contract
      promised: (<=/c 0)
      produced: 40
      in: (-> any/c (<=/c 0))           
      contract from: (interface for na-client)
      blaming: (interface for na-client)
       (assuming the contract is correct)
\end{verbatim}
The referenced contract is the compilation of the type of {\tt na-client}. The
definitive hint is ``{\tt blaming: (interface for na-client)}'' with the caveat
``{\tt (assuming the contract is correct)}.''

Assuming the rational programmer trusts the type of {\tt na-client}, the next
step is to inspect the {\tt layer} module and to equip it with type
annotations. The result is the \typecolor\ module in the middle, and {\tt
main}'s import is now re-directed there by {\tt (submod ".." layer)}. Running the
program again yields a new message:
\begin{verbatim}
    neg-abs: broke its own contract
      promised: (<=/c 0)
      produced: 10
      in: (-> any/c (<=/c 0))
      contract from: (interface for neg-abs)
      blaming: (interface for neg-abs)
       (assuming the contract is correct)
\end{verbatim}
The last step of the rational programmer is to assign types to the {\tt server}
module. At this point the type checker objects to the conjecture type of {\tt
neg-abs}---and the source of the impedance mismatch is found. How to fix it is a
separate question.

Like {\it homo economicus\/}, the rational programmer is an
approximation.  People don't behave purely rationally as economic
actors, and they also don't do so as software developers. The point is
not to deny the existence of ``lucky hunches'' programmers or
``tinkering works'' approaches and so on. It is also not to deny that
equipping entire modules with types represents an always feasible
approach.  But the concept of studying the idea of an economically
rational actor has produced benefits to political economics, and this
paper suggests that implementing and studying the rational programmer
will help language designers.
