%% -----------------------------------------------------------------------------

Trails and their properties provide the tools for a rigorous examination of
blame for Natural, Transient and Erasure. In line with the discussion so far,
the experiment software collects data to answer three initial questions for 
interesting debugging scenarios:
\begin{itemize}
\item[$Q_1$] Is blame useful in the context of Natural?

\item[$Q_2$] Is first blame useful in the context of Transient?

\item[$Q_3$] Is last blame useful in the context of Transient?

\end{itemize}

Furthermore, the experiment allows a comparison of the relative usefulness of blame
information:
\begin{itemize}
\item[$Q_*$] Is blame for X more useful than blame for Y (for X, Y in Natural, Transient, or Erasure)?
\end{itemize}


Table~\ref{fig:experiment-outline} summarizes how each question relates to
different kinds of trails/modes of the rational programmer. For example, experimental
question $Q_1$ asks whether blame is valuable for Natural and the experiment
uses the Natural blame and exception trails to answer it.

\begin{figure}[ht]
\center
{\begin{tabular}{l|c|c|c}
%% ---------------------------------------------------------------------------------------------------------------
                        & {\bf Natural}  & {\bf Transient} &  {\bf Erasure} \\ \hline 
%% ---------------------------------------------------------------------------------------------------------------
{\bf Blame}             &  $Q_1/Q_*$    &                  &                \\
{\bf First blame}       &               &     $Q_2/Q_*$    &                 \\
{\bf Last blame}        &               &     $Q_3/Q_*$    &                 \\
{\bf Exceptions}        &      $Q_1$    &     $Q_2/Q_3$    &      $Q_*$      \\
\end{tabular}}
  \caption{ Experimental questions and rational programmer modes.}
  \label{fig:experiment-outline}
\end{figure}

%% -----------------------------------------------------------------------------

In detail the answer to $Q_1$ demands a comparison of the success of the Natural
blame and Natural exception trails for all interesting debugging scenarios.  The
first step is to construct each mutant's scenario lattice and identify their
interesting debugging scenarios.  The experiment software extends the trails that start
from such roots according to the Natural-blame programmer.  If no scenarios can
be added to the trail, the experiment software checks whether the last scenario
of the trail type-checks or not. If it does, the software records that the
Natural-blame trail is successful; otherwise it is recorded as failing. The
process is repeated for the same roots again but for the other mode, Natural
exceptions.  Figure~\ref{fig:process} summarizes this experimental process for
one mode of the rational programmer and connects it with the mutations from
section~\ref{sec:mutate}.  After completing the experiment, the experiment
software reports the success/failure results of the trails to determine for each
root whether Natural blame is more useful than Natural exceptions.  Question
$Q_1$ has a positive answer if a root exists where the above is true because it
is evidence that there is at least one interesting scenario that the rational
programmer manages to debug with blame information.  The process is analogous
for $Q_2$ and $Q_3$, using the respective modes of the rational programmer.

For $Q_*$, the process is a bit more involved. Answering this question calls for
a comparison of the percentage of scenarios where one mode is more useful than
the other and the inverse.  For instance, to decide whether blame for Natural is
more useful than first blame for Transient, the software compares the percentage
of interesting scenarios where Natural blame is more successful than Transient
first blame with the percentage of interesting scenarios where Transient first
blame is more successful than Natural blame.  Finally these steps are repeated
one more time to compare Natural blame and Transient-last blame and get a
complete picture of the comparative usefulness of blame in the two semantics.

\begin{figure}
  \centering
  \includegraphics[scale=0.36]{./Images/process}
  \caption{The experimental process for one mode of the rational
  programmer}
  \label{fig:process}
\end{figure}

