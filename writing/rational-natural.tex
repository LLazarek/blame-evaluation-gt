%% -----------------------------------------------------------------------------

The Natural semantics assigns blame to exactly one boundary.  A blame assignment
has the following specific meaning: the typed module may make incorrect type
assumptions about the untyped module in its interface, or the correct
interface exposes a bug in the untyped module (or its dependencies). Our setup rules out the first
alternative (but see sec.~\ref{sec:conclusion}), and therefore the rational
programmer extends the trail to a scenario that swaps out the untyped
module for its typed counterpart.

% Formally, %% <-- MF: this is called a ``dangling def.'' and is considered bad style

%\mf{Technically a Natural blame trail could be a Natural exception trail. Hmph? }
%\mf{I added a sentence below this definition and the next to address this issue.}
%\cd{I revised a bit.}

Here is a rigorous definition. 
\begin{quote}
\it A \emph{Natural blame trail} is a sequence of scenarios $\conf_0,...\conf_n$ of
a program $\system$ such that for all $0 \leq i \leq n - 1$, $\conf_i \subset
\conf_{i+1}$ and
\[ \conf_{i+1} \setminus \conf_i = %
\left\{ %
\begin{tabular}{ll} %
$\{\blame{\system, \conf_i}\}$      & if $\conf_i$ produces blame \\ %
$\{\exception{\system, \conf_i}\}$  & otherwise \\ %
\end{tabular} \right. %
\]
where $\blame{\system, \conf}$ denotes the module (of $\system$) that $\conf$ blames,
and $\exception{\system, \conf}$ denotes the first untyped location in the stacktrace produced by $\conf$.
\end{quote}
Note how, in the absence of ``blame information'' in the narrow sense, 
this definition interprets ``blame information'' broadly, as in
any information from a failing run-time check.  
In particular, the definition rests on the two options for $\conf_{i+1} \setminus \conf_i$.
The first part reflects the rational programmer's use of blame to extend the
trail. However, blame may not be available in some scenarios. The second part
accounts for those scenarios that produce errors from the runtime system before
any impedance mismatch can be detected. For example, the fully untyped
configuration can only produce such an error, e.g. from {\tt length} receiving a
boolean. Exceptions from the runtime do not carry blame, so
the rational programmer proceeds using the accompanying stacktrace instead.

When the buggy untyped module of a program is replaced by the typed counterpart,
the type checker fails because this module causes the impedance mismatch. Hence, a
trail that ends at an ill-typed scenario successfully pinpoints the location of
the bug. 
\begin{quote}
\it A {\em Natural blame trail\/} $\conf_0,...\conf_n$ in a lattice $\lattice{\system}$ is
\emph{successful}\/ iff (the program for) its last scenario $\conf_n$ does not type check.  A Natural
blame trail $\conf_0,..,\conf_n$ in a lattice $\lattice{\system}$ is \emph{failing\/}
iff (the program for) $\conf_n$ type checks and the trail cannot be extended further.
\end{quote}
That is, failing Natural blame trails are those that end in a scenario that does not reveal the bug statically, yet also does not blame
an untyped module. Thus the rational programmer has
no further hints on how to continue the search for the bug.

While a successful Natural blame trail indicates that it 
pays off to heed blame assignments while debugging the trail's root, it does not answer whether
blame is a critical piece of the rational programmer's process.  For instance,
typing the top of a failed run-time type check's stack trace, dubbed the
location of the exception, might be as useful as typing the blamed one.

To account for this situation, a new mode of the Natural rational
programmer follows a migration process based entirely on exceptions.
\begin{quote}
\it A {\em Natural exception trail\/} is a sequence of scenarios $\conf_0,...\conf_n$ of a
program $\system$ such that for all $0 \leq i \leq n - 1$, $\conf_i \subset
\conf_{i+1}$ and $\conf_{i+1} \setminus \conf_i = \{\exception{\system, \conf_i}\}$.
\end{quote}
Using Natural exception trails, it becomes possible to factor out
``blame information'' in the narrow sense from the broad one of the
above definition. 


The definition of success for a Natural exception trail follows that for
a Natural blame trail.
Together, the definitions for the two modes allow the comparison of the usefulness of blame 
with that of mere exceptions for debugging a scenario in the context of Natural semantics.
\begin{quote}
\it 
  Given a program $\system$ and a root $\conf_0$ in $\lattice{\system}$,
  Natural blame is \emph{more useful\/} than Natural exceptions for
  debugging $\conf_0$ iff 
  the Natural blame trail 
  that starts at $\conf_0$ is successful while the Natural exception trail that
  starts at $\conf_0$ is failing.
\end{quote}
