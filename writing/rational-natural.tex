%% -----------------------------------------------------------------------------

The Natural semantics assigns blame to exactly one boundary.  A blame assignment
has the following specific meaning: the typed component may make incorrect type
assumptions about the untyped component in its interface, or the correct
interface exposes a bug in the untyped component (or its dependencies). Our setup rules out the first
alternative (but see section~\ref{sec:conclusion}), and therefore the rational
programmer extends the trail to a scenario that swaps out the untyped component
for its typed counterpart.

Formally,
\begin{quote}
\it A \emph{Natural blame trail} is a sequence of scenarios $\conf_0,...\conf_n$ of
a program $\system$ such that for all $0 \leq i \leq n - 1$, $\conf_i \subset
\conf_{i+1}$ and $\conf_{i+1} \setminus \conf_i =\{\blame{\system, \conf_i}\}$, where
$\blame{\system, \conf}$ denotes the component that $\conf$ blames.
\end{quote}

When the buggy untyped component of a program is replaced by the typed counterpart,
the type checker fails because this component causes the impedance mismatch. Hence, a
trail that ends at an ill-typed scenario successfully pinpoints the location of
the bug. 
\begin{quote}
\it A Natural blame trail $\conf_0,...\conf_n$ in a lattice $\lattice{\system}$ is
\emph{successful} iff (the program for) its last scenario $\conf_n$ does not type check.  A Natural
blame trail $\conf_0,..,\conf_n$ in a lattice $\lattice{\system}$ is \emph{failing}
iff (the program for) $\conf_n$ type checks and the trail cannot be extended further.
\end{quote}
That is, failing Natural blame trails are those that end in a scenario that does not reveal the bug statically, yet also does not blame
an untyped component. Thus the rational programmer gets
no hints from the gradual type system on how to continue the search for the bug.

While a successful Natural blame trail indicates that it 
pays off to heed blame assignments during debugging the trail's root, it does not answer whether
blame is a critical piece of the rational programmer's process.  For instance,
typing the top of a failed run time type check's stack trace, dubbed the
location of the exception, might be as useful as typing the blamed one.

To account for this situation, a new mode of the Natural rational
programmer follows a migration process based on exceptions:
\begin{quote}
\it A {\em Natural exception trail\/} is a sequence of scenarios $\conf_0,...\conf_n$ of a
program $\system$ such that for all $0 \leq i \leq n - 1$, $\conf_i \subset
\conf_{i+1}$ and $\conf_{i+1} \setminus \conf_i = \{\exception{\system, \conf_i}\}$
where $\exception{\system, \conf}$ denotes the location of the exception of $\conf$.
\end{quote}

The definition of success for a Natural exception trail follows that for
a Natural blame trail.
Together, the definitions for the two modes allow the comparison of the usefulness of blame 
with that of mere exceptions for debugging a scenario in the context of Natural semantics.
\begin{quote}
\it 
  Given a program $\system$ and a root $\conf_0$ in $\lattice{\system}$,
  Natural blame is \emph{more useful} than Natural exceptions for
  debugging $\conf_0$ iff 
  the Natural blame trail 
  that starts at $\conf_0$ is successful while the Natural exception trail that
  starts at $\conf_0$ is failing.
\end{quote}
