
In addition to successful and failure information for the various trails,  
our experiment  records the number of components a rational programmer 
has to type along each trail ($\lvert \conf_n \setminus \conf_0
\rvert$). We use this number as the metric for the effort 
of the rational programmer to debug an interesting scenario.  

A comparison of the effort of different modes of the rational programmer
can help shed some further light to the effectiveness of the
three gradual typing systems. For example, consider one debugging scenario.  
If, for instance, both  Natural blame and Transient first
blame trails are successful, the two modes of the 
rational programmer can compete to see which one
debugs the interesting scenario with less effort. In
general, if the effort distribution for a mode of the rational programmer
has a shorter tail and more
volume around smaller values compared to the effort distribution of another
mode, then this is evidence that the first mode is the more effective of the two.  

Finally, effort allows us to test whether the observed effectiveness 
of the rational programmer is an artifact of pure chance or not.
That is, we can compare the effort distribution for a mode of the
rational programmer with another mode that ignores error information entirely
and instead selects which component to type next randomly. 
Of course, since each mutant has a finite number of components, 
such random mode trails are always successful. At the same time though,
the Natural, Transient, and Erasure random modes exhibit a normal effort
distribution across scenarios that should be distinct from the effort distribution for all other modes of
the rational programmer if the effectiveness of the modes are not
coincidental.




