%% -----------------------------------------------------------------------------

In addition to success and failure information for the various trails, the
experiment records the number of modules a rational programmer has
to equip with types along each trail ($\lvert \conf_n \setminus \conf_0
\rvert$). This number can serve as an additional metric to compare modes.

Comparing the effort of different modes of the rational programmer can
illuminate the comparitive effectiveness of the three gradual typing
systems. If, for instance, both
Natural blame and Transient first blame trails are successful for the same scenario, the two modes of
the rational programmer can compete to see which one debugs the
scenario with less effort. In general, if the effort distribution for a mode of
the rational programmer has a shorter tail and more volume around smaller values
compared to the effort distribution of another mode, then the first mode is
probably the more effective of the two.

Measuring effort can also reveal whether the observed effectiveness of the
rational programmer is an artifact of pure chance.
In particular, the effort distribution for a mode of the rational programmer can be compared with that of
the random mode that ignores error information entirely and instead selects which
module to type randomly.  Since each mutant has a finite number of
modules, random mode trails are always successful. However, the random mode's effort
distribution will be thinly spread out across the range of trail lengths possible
in the set of debugging scenarios.
In contrast, the effort distribution of other modes should be quite different if their effectiveness is not coincidental.
