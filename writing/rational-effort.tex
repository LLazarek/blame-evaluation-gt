%% -----------------------------------------------------------------------------

In addition to success and failure information for the various trails, the
experiment software records the number of components a rational programmer has
to equip with types along each trail ($\lvert \conf_n \setminus \conf_0
\rvert$). This number can serve as a metric. 

A comparison of the effort of different modes of the rational programmer can
illuminate the comparison of the effectiveness of the three gradual typing
systems. For example, consider a debugging scenario.  If, for instance, both
Natural blame and Transient first blame trails are successful, the two modes of
the rational programmer can compete to see which one debugs the interesting
scenario with less effort. In general, if the effort distribution for a mode of
the rational programmer has a shorter tail and more volume around smaller values
compared to the effort distribution of another mode, then the first mode is
probably the more effective of the two.

Measuring effort can also tell whether the observed effectiveness of the
rational programmer is an artifact of pure chance or not.  Doing so means
comparing the effort distribution for a mode of the rational programmer with
another mode that ignores error information entirely and instead selects which
component to type next randomly.  Since each mutant has a finite number of
components, such random mode trails are naturally always successful. At the same
time though, the Natural, Transient, and Erasure random modes exhibit a normal
effort distribution across scenarios that should be distinct from the effort
distribution for all other modes of the rational programmer if the effectiveness
of the modes are not coincidental.




