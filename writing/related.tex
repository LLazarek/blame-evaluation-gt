\section{Related Work}

This paper builds on~\citet{lksfd-popl-2020}'s one-of-a-kind analysis of
blame for higher-order contracts. They do not spell out the notion of the
rational programmer but the basic ingredients of our technique derive
directly from theirs. Specifically, we adapt their empirical technique
to gradual typing which requires the development of custom mutators,
careful selection of debugging scenarios and generalization of the
technique to allow the comparison of different semantics. 

Besides the three semantics we consider, the literature on gradual typing
comes with a plethora of different approaches. For example, Pyret
(\shorturl{https://www.}{pyret.org}) treats function types same as Natural
and data types same as Erasure. The Forgetful~\cite{cl-icfp-2017} and
the Amnesic~\cite{gfd-oopsla-2019} semantics are similar  to Transient but
use wrappers instead of inlined checks.  Nom~\cite{mt-oopsla-2017} and
other \emph{concrete\/} semantics~\cite{wnlov-popl-2010, rsfbv-popl-2015,
rzv-ecoop-2015, rat-oopsla-2017} assume that every type maps to a unique
known tag and they perform type tag checks  for every value that crosses
from typed to untyped parts of a program or vice versa.  Monotonic
references~\cite{svctg-esop-2015} and the semantics~\cite{tlt-popl-2019,
etg-icfp-19, tt-scp-20, tgt-popl-18, tt-sas-17} derived with the
"turn-the-crank-get-a-POPL-paper" Abstracting Gradual Typing
technique~\cite{gct-popl-2016} are (optimized) variants of Natural for
different language features.  With the exception of Amnesic and Nom, all
the above semantics forego blame. However,  Nom's gradual types impose
significant restrictions on programmers, while Amnesic is a mere
theoretical construction and is not meant as a blueprint for an
implementation. 
 
