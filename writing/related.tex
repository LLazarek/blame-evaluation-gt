\section{Related Work}
\label{sec:related}

This paper builds on~\citet{lksfd-popl-2020}'s one-of-a-kind analysis of
blame for higher-order contracts. They do not spell out the notion of the
rational programmer but the basic ingredients of our technique derive
directly from theirs. Specifically, we adapt their empirical technique
to gradual typing which requires the development of custom mutators,
careful selection of debugging scenarios and generalization of the
technique to allow the comparison of different semantics.

The literature on gradual typing presents many semantics beyond the three
that we study, but only two additional blame strategies.
Pyret (\shorturl{https://www.}{pyret.org}) treats fixed-size data types same as Natural
and functions same as Transient. The Forgetful~\cite{cl-icfp-2017} and
the Amnesic~\cite{gfd-oopsla-2019} semantics are similar to Transient but
use wrappers instead of inlined checks.  Nom~\cite{mt-oopsla-2017} and
other \emph{concrete\/} semantics~\cite{wnlov-popl-2010, rsfbv-popl-2015,
rzv-ecoop-2015, rat-oopsla-2017} assume that every value comes with a type tag
and use tag checks to supervise the interactions between typed and untyped code.
Monotonic
references~\cite{svctg-esop-2015} and the semantics~\cite{tlt-popl-2019,
etg-icfp-19, tt-scp-20, tgt-popl-18, tt-sas-17} derived with the
Abstracting Gradual Typing
technique~\cite{gct-popl-2016} are (optimized) variants of Natural.
Among these representative semantics, only Amnesic and Nom present innovative
blame strategies.
Our study nevertheless excludes these strategies because Amnesic is a mere
theoretical construction and Nom imposes restrictions on untyped
code that would necessitate new data definitions across our benchmarks.
 
