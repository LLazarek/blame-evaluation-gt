\section{How to Sample the Experimental Space} 

The mutators for section~\ref{sec:mutate} generate 96,515,744 debugging scenarios; 
far too many to even identify the interesting ones among
them. Hence, we perform stratified uniform random sampling with two
levels of stratification to collect sufficiently many interesting
debugging scenarios for drawing statistically sound conclusions. 
The outcome of our sampling process is that we test each mode of the
rational programmer on 72,192 randomly selected interesting debugging
scenarios in total spread across both benchmarks and mutators.



As outlined in section~\ref{sec:rational}, our experimental questions
boil down whether a mode of the rational programmer is
as useful as another mode for all the interesting scenarios of a mutant $\system$.
Given the binary nature of this basic question, we can rely on random
sampling of interesting scenarios from $\lattice{\system}$ to obtain an
approximate answer.  Thus $\lattice{\system}$ becomes the first stratum
for our sampling; with a sufficiently large sample, we can obtain from the
sample an estimate of the usefulness of a mode for the entire lattice.
The estimate tells us with some confidence whether the mode is useful for all trails in the lattice (if all of the samples yield positive results) or none of them.
In fact, the estimate can say neither if some of our samples yield positive results and others yield negative ones; in that case we can say no more than that the mode is sometimes useful.
We select a sample size of 96 to obtain 95\% confidence and a 5\% margin of error for our estimates
(using standard statistical methods, the choice does not depend on the population size for sufficiently large populations).
For lattices that have less than 96 interesting scenarios, we select them all.

The second sampling stratum is the set of mutants. After all, our mutators
generate 16,800 mutants with interesting scenarios; still far too many to
explore even after sampling the lattice of each mutant. Thus we randomly
select from the mutants of each benchmark to obtain the largest
sample that keeps the experiment computationally feasible.
Our selection of mutants ensures that, per benchmark, the sample reflects the diversity of mutants with
respect to the mutators that generated them.  Specifically,  we sample 80
mutants with interesting scenarios per benchmark,  evenly-distributed
across all of the mutators that contribute mutants for the benchmark.


%% we use the following algorithm to select the 80 mutants evenly distributed across all of the mutators that do produce mutants of a benchmark:
%% 1. Treat each mutator as a bucket containing the mutants produced by that mutator
%% 2. Let the target-sample-size be 80
%% 3. Let the bucket-sample-size be target-sample-size/number-of-buckets
%% 4. For each bucket, do
%%    1. Randomly remove min(bucket-sample-size, mutants-in(bucket)) mutants from the bucket
%%    2. Set target-sample-size = target-sample-size - (the number of mutants removed)
%% 5. If target-sample-size is empty, done; else, go back to 3.

