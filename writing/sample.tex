\section{How to Sample the Experimental Space} 

The mutators for section~\ref{sec:mutate} generate 96,515,744 debugging scenarios; 
far too many to even identify the interesting ones among
them. Hence, we perform stratified uniform random sampling with two
levels of stratification to collect sufficiently many interesting
debugging scenarios for drawing statistically sound conclusions. 
The outcome of our sampling process is that we test each mode of the
rational programmer on 72192 randomly selected interesting debugging
scenarios in total spread across both benchmarks and mutators.



As outlined in section~\ref{sec:rational}, all our experimental questions
require that we decide whether a mode of the rational programmer is
always, sometimes or never useful for all the interesting scenarios of a
mutant $\system$. Assuming a normal distribution for the 
interesting scenarios in the lattice $\lattice{\system}$, we can obtain an
estimate answer with high confidence by randomly picking a sufficient
number of interesting scenarios $\lattice{\system}$. Thus
$\lattice{\system}$ becomes the first stratum for our sampling. For
95\% confidence, that is a 5\% margin, the sample size needs
to be just 95 out of $2^N$ scenarios in $\lattice{\system}$ where $N$ is the
number of components of $\system$.

The second sampling is the set of mutants. After all, our mutators
generate 16,800 mutants with
interesting scenarios; still far too many to explore even after sampling
the lattice of each mutant.  Unlike the lattice of scenarios of a mutant
, the set of mutants is not a coherent and finite population;  it is 
unclear how we can relate a sample of mutants to the
of all possible mutants. Thus we sacrifice the ability to generalize 
beyond the mutants we randomly select and opt for the largest random
sample that keeps the experiment computationally feasible. 
In particular,  we sample 80 mutants with interesting scenarios
per benchmark and for any benchmark, we ensure that our sample reflects the diversity of mutants with respect
to the mutators that generated them. Specifically, we select the 80 mutants evenly-distributed
across all of the mutators that contribute mutants for a benchmark.


%% we use the following algorithm to select the 80 mutants evenly distributed across all of the mutators that do produce mutants of a benchmark:
%% 1. Treat each mutator as a bucket containing the mutants produced by that mutator
%% 2. Let the target-sample-size be 80
%% 3. Let the bucket-sample-size be target-sample-size/number-of-buckets
%% 4. For each bucket, do
%%    1. Randomly remove min(bucket-sample-size, mutants-in(bucket)) mutants from the bucket
%%    2. Set target-sample-size = target-sample-size - (the number of mutants removed)
%% 5. If target-sample-size is empty, done; else, go back to 3.

