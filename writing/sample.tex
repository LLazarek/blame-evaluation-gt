\section{How to Sample the Experimental Space} 

The mutators from section~\ref{sec:mutate} generate 96,515,744 debugging
scenarios; far too many to even identify the interesting ones among them.
Furthermore this population is heterogeneous; different scenarios come
from different mutants and different mutants are the result of a
different mutator and benchmark pair. Hence, to make the experiment
computationally feasible and statistically precise, we perform stratified
uniform random sampling with three levels of stratification.  The first
two levels of strata group mutants first by benchmark and then by mutator. 
We randomly select from the mutants of
each benchmark ensuring that, per
benchmark, the sample reflects the diversity of mutants with respect to
the mutators that generated them.  Specifically,  we sample 80 mutants
with interesting scenarios per benchmark,  evenly-distributed across all
of the mutators that contribute mutants for the benchmark.  The third
level of strata for our sampling is the lattice of a single mutant and we draw 96 random
interesting scenarios from each lattice with replacement.

In summary, our sampling process takes advantage of ancillary information
to deliver a  sufficiently large representative set of scenarios.
Concretely, we test each
mode of the rational programmer on 72,192 randomly selected interesting
debugging scenarios spread across both benchmarks and mutators.
Furthermore, stratified random sampling allows us to derive precise conclusions
about the proportion of the whole population that satisfies a given trait 
from the sample despite the population's heterogeneity.
That property is exactly what we need to project comparisons about the
usefulness of various modes of the
rational programmer from the sample to the set of all interesting debugging
scenarios.  
 


%% we use the following algorithm to select the 80 mutants evenly distributed across all of the mutators that do produce mutants of a benchmark:
%% 1. Treat each mutator as a bucket containing the mutants produced by that mutator
%% 2. Let the target-sample-size be 80
%% 3. Let the bucket-sample-size be target-sample-size/number-of-buckets
%% 4. For each bucket, do
%%    1. Randomly remove min(bucket-sample-size, mutants-in(bucket)) mutants from the bucket
%%    2. Set target-sample-size = target-sample-size - (the number of mutants removed)
%% 5. If target-sample-size is empty, done; else, go back to 3.

