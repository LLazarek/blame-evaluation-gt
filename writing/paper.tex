\documentclass[acmsmall]{acmart}
\overfullrule=.1mm

\citestyle{acmauthoryear}

%%%%%%%%%%%%%%%%%%%%%%%%%%%%%%%%%%%%%%%%%%%%%%%%%%%%%%%%%%%%%%%%%%%%%%
%% Note: Authors migrating a paper from PACMPL format to traditional
%% SIGPLAN proceedings format must update the '\documentclass' and
%% topmatter commands above; see 'acmart-sigplanproc-template.tex'.
%%%%%%%%%%%%%%%%%%%%%%%%%%%%%%%%%%%%%%%%%%%%%%%%%%%%%%%%%%%%%%%%%%%%%%

%% Title information
\newcommand{\thetitle}{How to Evaluate Blame for Gradual Types}
\title{\thetitle}

%% Author information
%% Contents and number of authors suppressed with 'anonymous'.
%% Each author should be introduced by \author, followed by
%% \authornote (optional), \orcid (optional), \affiliation, and
%% \email.
%% An author may have multiple affiliations and/or emails; repeat the
%% appropriate command.
%% Many elements are not rendered, but should be provided for metadata
%% extraction tools.

\author{Lukas Lazarek}
\affiliation{%
  \institution{PLT @ Northwestern University}
  \city{Evanston}
  \state{Illinois}
  \country{USA}
}
\email{lukas.lazarek@eecs.northwestern.edu}

\author{Ben Greenman}
\orcid{0000-0001-7078-9287}
\affiliation{%
  \institution{PLT @ Northeastern University}
  \city{Boston}
  \state{Massachusetts}
  \country{USA}
}
\email{benjaminlgreenman@gmail.com}


\author{Matthias Felleisen}
\orcid{0000-0001-6678-1004}
\affiliation{%
  \institution{PLT @ Northeastern University}
  \city{Boston}
  \state{Massachusetts}
  \country{USA}
}
\email{matthias@ccs.neu.edu}

\author{Christos Dimoulas}
\orcid{0000-0002-9338-7034}
\affiliation{%
  \institution{PLT @ Northwestern University}
  \city{Evanston}
  \state{Illinois}
  \country{USA}
}
\email{chrdimo@northwestern.edu}
\renewcommand{\shortauthors}{Lazarek, Greenmanm Felleisen, Dimoulas}

\usepackage{alltt}
\usepackage{tcolorbox}
\usepackage{wrapfig}

%% Some recommended packages.
\usepackage{booktabs}   %% For formal tables:
                        %% http://ctan.org/pkg/booktabs
\usepackage{subcaption} %% For complex figures with subfigures/subcaptions
                        %% http://ctan.org/pkg/subcaption

\usepackage{listings}
\usepackage[shortcuts]{extdash}
\usepackage{xcolor}

\usepackage{upgreek}
\usepackage{multirow}

%\usepackage{draftwatermark}
%\SetWatermarkText{DRAFT}
%\SetWatermarkScale{1}


%%%%%%%%%%%%%%%%%%%%%%%%%%%%%%%%%%%%%%%%%%%%%%%%%%%%%%%%%%%%%%%%%%%%%%%%%%%%%%%%%%%%%%%%%%%%
%%%%%%%%%%%%%%%%%%%%%%%%%%%%%%%%%%%%%%%%%%%%%%%%%%%%%%%%%%%%%%%%%%%%%%%%%%%%%%%%%%%%%%%%%%%%
%%%%%%%%%%%%%%%%%%%%%%%%%%%%%%           PREAMBLE       %%%%%%%%%%%%%%%%%%%%%%%%%%%%%%%%%%%% 
%%%%%%%%%%%%%%%%%%%%%%%%%%%%%%%%%%%%%%%%%%%%%%%%%%%%%%%%%%%%%%%%%%%%%%%%%%%%%%%%%%%%%%%%%%%%
%%%%%%%%%%%%%%%%%%%%%%%%%%%%%%%%%%%%%%%%%%%%%%%%%%%%%%%%%%%%%%%%%%%%%%%%%%%%%%%%%%%%%%%%%%%%

% acmart formatting
\settopmatter{printfolios=true}
\overfullrule=1mm

% techreport title
\newcommand{\techreport}{supplementary material}

% short url formatting
\newcommand{\shorturl}[2]{\href{#1#2}{\texttt{#2}}}

%%%%%%%%%%%%%%%%%%%%%%%%%%%%%%%%%%%%%%%%%%%%%%%%%%%%%%%%%%%%%%%%%%%%%%%%%%%%%%%%%%%%%%%%%%%%
%%%%%%%%%%%%%%%%%%%%%%%%%%%%%%%%%%%%%%%%%%%%%%%%%%%%%%%%%%%%%%%%%%%%%%%%%%%%%%%%%%%%%%%%%%%%
%%%%%%%%%%%%%%%%%%%%%%%%%%%%%%%%%%%%%%%%%%%%%%%%%%%%%%%%%%%%%%%%%%%%%%%%%%%%%%%%%%%%%%%%%%%%


%% lattice

\newcommand{\system}{\ensuremath{P}}



\newcommand{\component}{\ensuremath{c}}

\newcommand{\latticeL}{\mathcal{L}}

\newcommand{\lattice}[1]{\ensuremath{\latticeL\llbracket#1\rrbracket}}

\newcommand{\standardlattice}{\lattice{\system}{\kmap}}

\newcommand{\conf}{\ensuremath{s}}

\newcommand{\metric}{\ensuremath{\leq_{\latticeL}^{X}}}

\newcommand{\set}[1]{\ensuremath{\bar{#1}}}

\newcommand{\size}[1]{\ensuremath{\mid #1 \mid}}

\newcommand{\blame}[1]{\ensuremath{blame\llbracket#1\rrbracket}}

\newcommand{\mblame}[1]{\ensuremath{multiblame\llbracket#1\rrbracket}}



\newcommand{\faulty}[1]{\ensuremath{FAULTY\llbracket#1\rrbracket}}

\newcommand{\random}[1]{\ensuremath{random\llbracket#1\rrbracket}}

\newcommand{\error}[1]{\ensuremath{error\llbracket#1\rrbracket}}

\newcommand{\effort}[1]{\ensuremath{EFFORT\llbracket#1\rrbracket}}

\newcommand{\first}[1]{\ensuremath{first\llbracket#1\rrbracket}}

\newcommand{\last}[1]{\ensuremath{last\llbracket#1\rrbracket}}


\newcommand{\modem}{\ensuremath{S}}
\newcommand{\mode}[1]{\ensuremath{\modem\llbracket#1\rrbracket}}





%% misc

\newcommand{\setsize}[1]{\left|#1\right|}

\newcommand{\codesize}{\small}

\newcommand{\codecolor}{black}

\let\OldTexttt\texttt
\renewcommand{\texttt}[1]{\OldTexttt{\color{\codecolor}{#1}}}

\let\Oldtt\tt
\renewcommand{\tt}{\Oldtt \color{\codecolor} }

\newcommand{\todo}[1]{{\textcolor{red}{TODO}\{\textcolor{gray}{#1}\}}}

\newenvironment{code}{\codesize \color{\codecolor} \begin{alltt}}{\end{alltt}}


%% %%%%%%%%%%%%%%%%%%%%%%%%%%%%%%%%%%%%%%%%%%%%%%%%%%%%%%%%%%%%%%%%%%%%%%%%%%%%%
%% for leaving margin notes in the paper write
%% \yourinitials{...} 

\def\notes#1{\expandafter\def\csname#1\endcsname##1{\marginpar{\raggedright\tiny $\bullet$ #1 says: ##1}}}
\notes{mf}
\notes{cd}
\notes{ll}
\notes{bg}


\acmConference[]{}{}{}
\acmYear{}
\acmISBN{} % \acmISBN{978-x-xxxx-xxxx-x/YY/MM}
\acmDOI{} % \acmDOI{10.1145/nnnnnnn.nnnnnnn}
\startPage{1}

%% Copyright information
%% Supplied to authors (based on authors' rights management selection;
%% see authors.acm.org) by publisher for camera-ready submission;
%% use 'none' for review submission.
%% \setcopyright{none}

%% Bibliography style
\bibliographystyle{ACM-Reference-Format}

\begin{document}

%% Abstract
%% Note: \begin{abstract}...\end{abstract} environment must come
%% before \maketitle command
\begin{abstract}
Programming language theoreticians develop blame assignment systems and
  prove blame theorems for gradually typed programming languages.
  Practical implementations of gradual typing almost completely ignore the
  idea of blame assignment.  This contrast raises the question whether
  blame provides any value to the working programmer and poses
  the challenge of how to evaluate the effectiveness of blame assignment
  strategies. This paper contributes (1) the first evaluation method for
  blame assignment strategies and (2) the results from applying it to
  three different semantics for gradual typing. These results cast doubt on
  the theoretical effectiveness of blame in gradual typing. In most scenarios, strategies
  with imprecise blame assignment are as helpful to a rationally acting
  programmer as strategies with provably correct blame.
\end{abstract}



% CCS concepts: Program specifications < Program reasoning < Semantics and Reasoning < Theory of Computation
% CCS concepts: Empirical software validation < Software verification and validation < Software creation and management < Software and its engineering

% potential choices decided against: pre- and post-conditions, assertions, software testing and debugging

%\begin{CCSXML}
%<ccs2012>
%<concept>
%<concept_id></concept_id>
%<concept_desc></concept_desc>
%<concept_significance></concept_significance>
%</concept>
%<concept>
%<concept_id></concept_id>
%<concept_desc></concept_desc>
%<concept_significance></concept_significance>
%</concept>
%</ccs2012>
%\end{CCSXML}

%\ccsdesc[]{}
%\ccsdesc[]{}


%% Keywords
%% comma separated list
%\keywords{}  %% \keywords are mandatory in final camera-ready submission


%% \maketitle
%% Note: \maketitle command must come after title commands, author
%% commands, abstract environment, Computing Classification System
%% environment and commands, and keywords command.
\maketitle

%% -----------------------------------------------------------------------------
\def\sec#1#2{\section{#2} \label{sec:#1} \input{#1.tex}}

\sec{introduction}{Does Blame Matter}
\def\spacrr{.75cm}
\begin{figure*}[bh] \footnotesize

\begin{minipage}[b]{6.1cm}
\begin{module}{types : typed/racket}\typecolor
(provide Entry Entries)

(define-type Entry
  (Pairof Symbol String))

(define-type Entries
  (Listof Entry))
\end{module}

\vspace\spacrr

\begin{module}{typed-pack-lib : typed/racket}\typecolor
(provide typed-pack)

(require types)
(require/typed pack-lib
  [pack (-> JSON Entries)])

(define typed-pack pack) 
\end{module}

\vspace\spacrr

\begin{module}{crypto-pack-lib : typed/racket}\typecolor
(provide crypto-pack) 

(require types)
(require/typed pack-lib
 [pack (-> JSON Entries)])

(: crypto-pack (-> JSON Entries))
(define (crypto-pack d)
  (pack (encrypt d _ _ _)))
\end{module}\end{minipage}\hfil\begin{minipage}[b]{6.1cm}
\begin{module}{pack-lib : racket}\dyncolor
(provide pack _ _ _)

_ _ _  dependencies     _ _ _
_ _ _  and definitions  _ _ _

(define (pack d)
  ;; process JSON data and 
  ;; package as a dictionary
  ;; (association list)
  (make-hash _ _ _) ;; BUG!)
\end{module}
\begin{module}{client : racket}\dyncolor
(require json)
(require typed-pack-lib)
(reqired crypto-pack-lib)

_ _ _ other dependencies _ _ _
_ _ _ and definitions    _ _ _
          
;; read data from files, pack 
;; and share securely 

(define public-data 
  (typed-pack 
   (read-json 
    "public-records")))

(define secret-data 
  (crypto-pack 
   (read-json 
    "medical-records")))
 
 _ _ _ rest of client _ _ _
\end{module}
\end{minipage}


\caption{One mix-typed program, three interpretations} \label{fig:example}
\end{figure*}

\sec{landscape}   {How to Think About Three Blame Systems}
\sec{why-rational}{What Is a Rational Programmer}
\sec{challenges}  {Why It Is Hard to Evaluate Blame}
\sec{rational}    {How to Make Comparable Rational Programmers}
\sec{mutate}      {How to Make Lots of Mistakes}
\sec{sample}      {How to Make the Experimental Space Tractable}
\sec{results}     {What Are the Outcomes of the Experiment}
\sec{discussion}  {What Can Programmers Learn}
\sec{related}     {Who has Worked on this Problem Before}
\sec{conclusion}  {What to Do Next}
%% -----------------------------------------------------------------------------

%% Acknowledgments
\begin{acks}                 
Felleisen and Greenman were partly supported by NSF grant SHF
1763922. Greenman also received support from a CRA-NSF Post-doctoral
fellowship grant. Dimoulas and Felleisen wish to thank Max New for
extensive discussions of the early ideas in this paper. Thanks also to
Robby Findler and Northwestern PLT for their valuable feedback at
various stages of this work.
\end{acks}


%% Bibliography
\bibliography{cs}

\end{document}
