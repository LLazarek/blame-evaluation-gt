%% For double-blind review submission, w/o CCS and ACM Reference (max submission space)
\documentclass[acmsmall,screen,anonymous,nonacm]{acmart}\settopmatter{printfolios=true,printccs=true,printacmref=true}
%% For double-blind review submission, w/ CCS and ACM Reference
%%\documentclass[acmsmall,review,anonymous]{acmart}\settopmatter{printfolios=true}
%% For single-blind review submission, w/o CCS and ACM Reference (max submission space)
%\documentclass[acmsmall,review]{acmart}\settopmatter{printfolios=true,printccs=false,printacmref=false}
%% For single-blind review submission, w/ CCS and ACM Reference
%\documentclass[acmsmall,review]{acmart}\settopmatter{printfolios=true}
%% For final camera-ready submission, w/ required CCS and ACM Reference
%\documentclass[acmsmall]{acmart}\settopmatter{}

%%%%%%%%%%%%%%%%%%%%%%%%%%%%%%%%%%%%%%%%%%%%%%%%%%%%%%%%%%%%%%%%%%%%%%
%% Note: Authors migrating a paper from PACMPL format to traditional
%% SIGPLAN proceedings format must update the '\documentclass' and
%% topmatter commands above; see 'acmart-sigplanproc-template.tex'.
%%%%%%%%%%%%%%%%%%%%%%%%%%%%%%%%%%%%%%%%%%%%%%%%%%%%%%%%%%%%%%%%%%%%%%

%% Title information
\newcommand{\thetitle}{How to Evaluate Blame for Gradual Types}
\title{\thetitle}

%% Author information
%% Contents and number of authors suppressed with 'anonymous'.
%% Each author should be introduced by \author, followed by
%% \authornote (optional), \orcid (optional), \affiliation, and
%% \email.
%% An author may have multiple affiliations and/or emails; repeat the
%% appropriate command.
%% Many elements are not rendered, but should be provided for metadata
%% extraction tools.

\author{Nobody In Particular}
% \affiliation{\institution{PLT} \city{Evanston} \country{Illinois} }
\affiliation{\relax} % \institution{PLT} \city{Evanston} \country{Illinois} }
% \email{chrdim@northwestern.edu}          %% \email is recommended



\usepackage{alltt}
\usepackage{wrapfig}

%% Some recommended packages.
\usepackage{booktabs}   %% For formal tables:
                        %% http://ctan.org/pkg/booktabs
\usepackage{subcaption} %% For complex figures with subfigures/subcaptions
                        %% http://ctan.org/pkg/subcaption

\usepackage{listings}
\usepackage[shortcuts]{extdash}
\usepackage{xcolor}

\usepackage{upgreek}

%\usepackage{draftwatermark}
%\SetWatermarkText{DRAFT}
%\SetWatermarkScale{1}


%%%%%%%%%%%%%%%%%%%%%%%%%%%%%%%%%%%%%%%%%%%%%%%%%%%%%%%%%%%%%%%%%%%%%%%%%%%%%%%%%%%%%%%%%%%%
%%%%%%%%%%%%%%%%%%%%%%%%%%%%%%%%%%%%%%%%%%%%%%%%%%%%%%%%%%%%%%%%%%%%%%%%%%%%%%%%%%%%%%%%%%%%
%%%%%%%%%%%%%%%%%%%%%%%%%%%%%%           PREAMBLE       %%%%%%%%%%%%%%%%%%%%%%%%%%%%%%%%%%%% 
%%%%%%%%%%%%%%%%%%%%%%%%%%%%%%%%%%%%%%%%%%%%%%%%%%%%%%%%%%%%%%%%%%%%%%%%%%%%%%%%%%%%%%%%%%%%
%%%%%%%%%%%%%%%%%%%%%%%%%%%%%%%%%%%%%%%%%%%%%%%%%%%%%%%%%%%%%%%%%%%%%%%%%%%%%%%%%%%%%%%%%%%%

% acmart formatting
\settopmatter{printfolios=true}

% techreport title
\newcommand{\techreport}{supplementary material}

% short url formatting
\newcommand{\shorturl}[2]{\href{#1#2}{\texttt{#2}}}

%%%%%%%%%%%%%%%%%%%%%%%%%%%%%%%%%%%%%%%%%%%%%%%%%%%%%%%%%%%%%%%%%%%%%%%%%%%%%%%%%%%%%%%%%%%%
%%%%%%%%%%%%%%%%%%%%%%%%%%%%%%%%%%%%%%%%%%%%%%%%%%%%%%%%%%%%%%%%%%%%%%%%%%%%%%%%%%%%%%%%%%%%
%%%%%%%%%%%%%%%%%%%%%%%%%%%%%%%%%%%%%%%%%%%%%%%%%%%%%%%%%%%%%%%%%%%%%%%%%%%%%%%%%%%%%%%%%%%%


%% lattice

\newcommand{\system}{\ensuremath{P}}



\newcommand{\component}{\ensuremath{c}}

\newcommand{\latticeL}{\mathcal{L}}

\newcommand{\lattice}[1]{\ensuremath{\latticeL\llbracket#1\rrbracket}}

\newcommand{\standardlattice}{\lattice{\system}{\kmap}}

\newcommand{\conf}{\ensuremath{s}}

\newcommand{\metric}{\ensuremath{\leq_{\latticeL}^{X}}}

\newcommand{\set}[1]{\ensuremath{\bar{#1}}}

\newcommand{\size}[1]{\ensuremath{\mid #1 \mid}}

\newcommand{\blame}[1]{\ensuremath{blame\llbracket#1\rrbracket}}

\newcommand{\exception}[1]{\ensuremath{exception\llbracket#1\rrbracket}}


\newcommand{\mblame}[1]{\ensuremath{multiblame\llbracket#1\rrbracket}}



\newcommand{\faulty}[1]{\ensuremath{FAULTY\llbracket#1\rrbracket}}

\newcommand{\random}[1]{\ensuremath{random\llbracket#1\rrbracket}}

\newcommand{\error}[1]{\ensuremath{error\llbracket#1\rrbracket}}

\newcommand{\effort}[1]{\ensuremath{EFFORT\llbracket#1\rrbracket}}

\newcommand{\first}[1]{\ensuremath{first\llbracket#1\rrbracket}}

\newcommand{\last}[1]{\ensuremath{last\llbracket#1\rrbracket}}


\newcommand{\modem}{\ensuremath{S}}
\newcommand{\mode}[1]{\ensuremath{\modem\llbracket#1\rrbracket}}





%% misc

\newcommand{\setsize}[1]{\left|#1\right|}

\newcommand{\codesize}{\small}

\newcommand{\codecolor}{black}

\let\OldTexttt\texttt
\renewcommand{\texttt}[1]{\OldTexttt{\color{\codecolor}{#1}}}

\let\Oldtt\tt
\renewcommand{\tt}{\Oldtt \color{\codecolor} }

\newcommand{\todo}[1]{{\textcolor{red}{TODO}\{\textcolor{gray}{#1}\}}}

\newenvironment{code}{\codesize \color{\codecolor} \begin{alltt}}{\end{alltt}}


%% %%%%%%%%%%%%%%%%%%%%%%%%%%%%%%%%%%%%%%%%%%%%%%%%%%%%%%%%%%%%%%%%%%%%%%%%%%%%%
%% for leaving margin notes in the paper write
%% \yourinitials{...} 

\def\notes#1{\expandafter\def\csname#1\endcsname##1{\marginpar{\textcolor{red}{\raggedright\tiny $\bullet$ #1 says: ##1}}}}
\notes{mf}
\notes{cd}
\notes{ll}
\notes{bg}


%% \begin{module}{name : lang}{color} ... \end{module}
\newenvironment{module}[2]{
\begin{minipage}[t]{6cm}
\begin{modulecolorbox}[width=6.0cm]{#2}{#1}
\begin{alltt}}{
\end{alltt}
\end{modulecolorbox}
\end{minipage}}

%% \begin{modulecolorbox}[options-for-tcolorbox]{color}{title} ... \end{modulecolorbox}
\newtcolorbox{modulecolorbox}[3][]{
  colframe = #2!25,
  colback  = #2!10,
  coltitle = #2!20!black,  
  title    = {#3},
  #1,
}


%% space between layers of boxes 
\def\spacr{.25cm}

%% colors
\def\typecolor{blue}
\def\dyncolor{red}



%% Journal information
%% Supplied to authors by publisher for camera-ready submission;
%% use defaults for review submission.
\startPage{1}

%% Copyright information
%% Supplied to authors (based on authors' rights management selection;
%% see authors.acm.org) by publisher for camera-ready submission;
%% use 'none' for review submission.
%% \setcopyright{none}
%\setcopyright{acmcopyright}
%\setcopyright{acmlicensed}
%\setcopyright{rightsretained}
%\copyrightyear{2018}           %% If different from \acmYear

%% Bibliography style
\bibliographystyle{ACM-Reference-Format}
%% Citation style
%% Note: author/year citations are required for papers published as an
%% issue of PACMPL.
\citestyle{acmauthoryear}   %% For author/year citations

%%%
\setcopyright{none}
\acmPrice{1}
\acmDOI{}
\acmYear{}
\copyrightyear{}
\acmJournal{pacmpl}
\acmVolume{1}
\acmNumber{1}
\acmArticle{1}
\acmMonth{1}


\begin{document}

%% Abstract
%% Note: \begin{abstract}...\end{abstract} environment must come
%% before \maketitle command
\begin{abstract}
Programming language theoreticians develop blame assignment systems
and prove blame theorems for gradually typed programming languages.
Practical implementations of gradual typing almost completely ignore
the idea of blame assignment.  This contrast raises the question
whether blame assignment provides any value to the working programmer
and poses the challenge of how to evaluate the effectiveness of blame
assignment strategies. This paper contributes (1) the first evaluation
method for blame assignment strategies and (2) the results from applying
it to three different gradual type systems. \mf{This sentence must be checked
before final submission.} These results confirm that blame assignment
is helpful to a rationally acting programmer.
\end{abstract}



% CCS concepts: Program specifications < Program reasoning < Semantics and Reasoning < Theory of Computation
% CCS concepts: Empirical software validation < Software verification and validation < Software creation and management < Software and its engineering

% potential choices decided against: pre- and post-conditions, assertions, software testing and debugging

%\begin{CCSXML}
%<ccs2012>
%<concept>
%<concept_id></concept_id>
%<concept_desc></concept_desc>
%<concept_significance></concept_significance>
%</concept>
%<concept>
%<concept_id></concept_id>
%<concept_desc></concept_desc>
%<concept_significance></concept_significance>
%</concept>
%</ccs2012>
%\end{CCSXML}

%\ccsdesc[]{}
%\ccsdesc[]{}


%% Keywords
%% comma separated list
%\keywords{}  %% \keywords are mandatory in final camera-ready submission


%% \maketitle
%% Note: \maketitle command must come after title commands, author
%% commands, abstract environment, Computing Classification System
%% environment and commands, and keywords command.
\maketitle


Theoreticians of gradual typing have focused on blame theorems from the very
beginning~\cite{mf-toplas-2009, tf-dls-2006}. ``Well-typed
[components]\footnote{The original authors got this word wrong.} can't be
blamed'' turned the theorem into a slogan~\cite{wf-esop-2009}. Academic systems
(Reticulated Python~\cite{vsc-dls-2019, vss-popl-2017, vksb-dls-2014} and Typed
Racket~\cite{tf-dls-2006,tf-popl-2008,tfffgksst-snapl-2017,tf-icfp-2010}) come
with sophisticated blame assignment strategies. Their academic creators embrace
the idea that blame can help practicing programmers find impedance mismatches
between the types they added to a software system and the behavior of the
remaining untyped components.

Industrial implementors of gradual typing systems have almost
completely ignored blame assignment.  Systems such as Flow, Hack, or
TypeScript\footnote{See \url{https://flow.org},
\url{https://hacklang.org}, and \url{https://www.typescriptlang.org},
respectively.} exploit types for IDE actions and for finding typos in
code. Then their compilers remove types and rely on the built-in
safety checks of the underlying language to catch any problems.

This contrast between theory and applications of gradual typing raises the question 
\begin{quote}
 \it
 whether blame assignment adds any value to a gradually typed language,
 especially for the benefit of the working programmer.
\end{quote}
Given the long-standing academic interest in blame and its complete absence in
industrial systems, it comes as an even bigger surprise that the research
literature and the industrial blog world do not discuss any possible answers.
Instead, when language designers make relevant decisions, they seem to answer it
one way or another without any scientific justification. Then again, the
community has thus far failed to offer a method for evaluating blame assignment.

This paper's {\em first contribution\/} is {\em a method for evaluating the
effectiveness of blame assignment strategies\/} in the gradual typing world.
The top-level innovation is the idea of a {\em rational programmer\/}, that is,
a programmer that acts only in response to available information (see
sec.~\ref{sec:why-rational}). In the case of an impedance mismatch, the
available information consists of the error message and the current state of the
program. The rational programmer can hence use the former to change the
latter---and this systematic, information-driven process can be
implemented.\footnote{\citet{cc-snapl-19} use this tracing method to localize
type errors in inference-based gradual typing systems.}  Implementing it means
overcoming major challenges: injecting representative impedance mismatches;
putting the various kinds of error information to comparable use; and sampling
the huge space of possibilities (see sec.~\ref{sec:challenges}).

The paper's {\it second contribution\/} is a set of {\em results from applying
the evaluation method to three distinct blame assignment strategies and
approximately 72,200 cases\/} (see sec.~\ref{sec:results}): (1) {\it
Transient\/}, i.e. Reticulated's tracking of typed/untyped boundaries; (2) {\it
Natural\/}, i.e. the use of higher-order contract system and its blame
assignment~\cite{ff-icfp-2002}; and (3) {\it Erasure\/}, i.e. the approach of
industrial systems, which forgo blame in favor of error messages from the safety
checks of the underlying language.---The results (see
sec.~\ref{sec:discussion}) are at least somewhat surprising.  In principle
they validate the conjectures behind the work of theoreticians.  First, a good
blame assignment strategy helps with the search for impedance mismatches between
the specified types and the behavior of untyped components.  Second, Natural's
wrapper-based blame tracking is more useful than Transient's ``collective
blame'' tracking algorithm, which in turn is superior to Erasure. {\em But\/},
the application of the method also indicates problems with the expectations of
theoreticians. The first problem is that the cost of Transient's blame can be
huge. The second one concerns the difference between the methods; neither is
{\it Natural\/} vastly better than {\it Transient\/} nor are the two blame
assignment methods clearly preferable to {\it Erasure\/}.  In turn, these
problems suggests that, on one hand, the existing theory does not predict
practice properly, and on the other hand, the existing practice may need to find
additional ways to collect data.




 
%% -----------------------------------------------------------------------------

Implementing the method of the preceding section poses three challenges.  The
first concerns the comparison of the effect of blame on the rational programmer
across three different mechanisms; the second challenge is about finding a large
number of representative debugging scenarios; and the third is the resulting
huge space of possibilities. A coincidental challenge is the disparity of the
implementations of gradually typed languages. To eliminate this variable, the
authors use Racket, which is thus far the only language in which all three major
semantic variants are available in a robust and comparable manner: Typed Racket
implements Natural, Shallow Racket~\citep{ttt21} Transient, and plain Racket
Erasure.

The {\em first challenge\/} stems from the differences between the blame
assignment mechanisms of the three semantic variants.  While Natural assigns
blame to {\em one\/} component, Transient assigns blame to a sequence of
components. The Erasure semantics does not blame components {\it per se\/}, but
it comes with an exception location and a stack trace, which implicitly suggest
fixes.  Each strategy triggers different reactions by the rational programmer
(and real ones, too).

One way to reconcile these differences is to {\em equip the rational
programmer with modes \/} that represent the different types of
information the rational programmer takes into account when debugging a
scenario. Intuitively, different
blame strategies correspond to different modes of operation.
For instance, one Transient mode may assign types to the oldest element of a
blame sequence because it corresponds to the
earliest point in the execution that can discover an impedance
mismatch.  Another mode may opt to treat the sequence as a stack and add
types to its newest element.  If both modes are equally successful in
locating an impedance mismatch, measuring the rational programmer's effort with each mode
may answer which is the most effective.

However, attributing the success of the rational programmer to this or
that blame mechanism demands a careful analysis of the interplay between blame
and the run-time checks of each gradual type system. When a check fails,
the Natural and Transient semantics assign blame instead of using the
information in the exception from the run-time check. But, the
exception information may be as useful to the rational
programmer as a blame assignment. If this is the case, then blame {\em per se\/} may not play a critical
role for the rational programmer. Indeed, precisely because they do not
account for such confounding factors, \citet{lksfd-popl-2020} cannot draw any
conclusions about blame specifically, despite advertisements to the
opposite. Their experiment may conclude only that so-called blame-shifting
works, but they cannot attribute this conclusion to blame alone. 

Modes offer a uniform way to compare the different semantics and isolate blame
from the effect of the semantics' run-time checks. Specifically, the rational
programmer comes with a blame mode and an exception mode for Natural and
Transient. If the blame mode succeeds in debugging a program while the exception
one fails, it is safe to conclude that blame is indeed beneficial for the
rational programmer. Put differently, the exception mode serves as the {\em
baseline\/} for blame's value within a given semantics; if the programmer in
this mode performs as well or better than the blame one, a blame assignment
mechanism might be useless.

An experiment must also rule out that the usefulness of blame assignment is sheer
luck.  Hence, a completely random mode provides yet another necessary baseline.

The {\em second challenge\/} is to find a representative collection of programs
with impedance mismatches.\footnote{\citet{cc-oopsla-20} created a collection of
Reticulated Python programs to evaluate their technique of fixing mistakes in
type annotations. Their collection does not come with type-level mistakes in the
code itself.}  The impedance mismatches must represent mistakes that programmers accidentally
create and that the run-time checks of academic systems catch. In other words,
the experiment calls for a collection of mistakes in mixed-typed programs that is
representative of those ``in the wild.''  Unfortunately no such collection
exists, and with good reason. The kind of mistakes needed are typically detected
by unit or integration tests; even if it takes some time to find their sources,
these mistakes do not make it into code repositories with appropriate commit
messages.

An alternative is to {\em generate a corpus of 
mistakes \/} using mutation analysis~\citep{lipton1971fault, demillo1978hints,
jia2011analysis}, but conventional mutation analysis is useless.  Mutation
analysis traditionally aims to inject bugs that challenge test suites, and it
discards those that yield ill-typed mutants as \emph{incompetent}. Indeed,
mutation analysis frameworks are fine-tuned to avoid them, and yet, it is
precisely those mutators that are needed for evaluating blame assignment strategies.

Based on a related experience, \citet{gw-mutation} write,
``existing mutation frameworks \ldots\ do not generate the kinds of mutations
needed to best evaluate type annotations'' and, worse, ``it is surprisingly
difficult to come up with mutants that actually describe subtle type faults.''
While the goal of their work---to evaluate the quality of types in
Python---is unrelated to blame, the mechanism is related. And their
judgment confirms the experience of the authors. 

An experimental analysis of blame needs a mostly new set of mutators.
Roughly speaking, the new mutators inject type errors into fully typed programs.
Applying such a mutator to any typed component produces a mutated component.  A
debugging scenario results from removing the types from the mutated
component. For the design of such mutators, the authors relied on their own
extensive programming experience though not without discovering a major pitfall:
some of their original mutators systematically produced programs that
immediately revealed the source of the impedance mismatch.  All of the remaining
ones yield {\em interesting debugging scenarios\/} (see
sec.~\ref{sub:mutate-interesting}).

The {\em third challenge\/} is the explosive number of debugging scenarios that
result from the combination of mutation-based scenario generation and mode-based
analysis. All three factors---three different gradual typing systems, the large
number of mutants, and the number of debugging modes---contribute possibly
useful experimental data in a multiplicative manner. Hence, carried out naively,
the experiment would demand an infeasible amount of computational
resources.  A practical execution has no option but to {\em sample the space of
scenarios\/}, carefully ensuring reproducibility.

The next three sections explain how to overcome the three challenges
in detail. 
   
\section{How to Bring Natural, Transient and Erasure Under the Same Roof} \label{sec:landscape}


   By example; adjusted version of the one in OOPSLA sec 2

  \begin{itemize}

  \item Why Blame Assignments Differ.

  \item How Blame Assignments Differ.

  \item A space of GT system design. In terms of error reporting, we identify two axes of GT system
        design: where checking occurs and whether blame is tracked.
    

  \end{itemize}



\subsection{Natural Typed Racket}

Typed Racket extends the untyped Racket language with \emph{natural}\/ gradual
 types~\cite{tf-popl-2008,tfffgksst-snapl-2017}.
Natural types guarantee a strict separation between typed and untyped code;
 informally, an untyped value can interact with a typed context only if the
 value behaves in a well-typed way.
Typed Racket implements this separation by compiling types to higher-order
 contracts that guard all boundaries to untyped code.
Any higher-order typed value that exits a boundary receives a protective
 wrapper, and any untyped value that enters must pass a deep run-time type
 check.
These contract checks help predict the behavior of typed code, but
 may greatly slow down a program.


\subsection{Transient Typed Racket}

Transient gradual types ensure that typed code cannot get stuck on a primitive
 operation.
This assurance can be implemented without higher-order
 contracts~\cite{vss-popl-2017}, and therefore offers a modicum of type
 soundness for a seemingly-low implementation and performance cost.

For this experiment, we modified Typed Racket to bring transient and natural
 under the same roof.
Our version of the library is parameterized by a \emph{type-enforcement strategy}
 to allow reuse of common components.
Under the \emph{natural} strategy, our Typed Racket is identical to the latest release.
By contrast, the \emph{transient} strategy re-uses the static type checker and
 much of the compiler, but weakens the run-time meaning of types in two ways.

The first major ingredient in transient Typed Racket is a new interpretation
 of types as first-order shape checks.
Each shape-check enforces a type constructor.
For certain types, such as {\tt String} or {\tt (Vectorof Byte)}, transient can
 re-use a predicate from the underlying Racket language.
Other types require custom predicates; for example the type {\tt Index} of
 valid array indices allows a range of positive exact integers.
Many custom predicates run in constant time, but some are more involved.
One linear-time check is for {\tt (Listof T)} for any element type {\tt T},
 because it must distinguish a null-terminated list of pairs from improper
 list-like structures.
The key point, however, is that no transient check relies on
 higher-order contract wrappers.

Second, transient adds a new pass to the compiler that rewrites all typed code.
Rewriting changes all elimination forms to shape-check the result.
For example, if the variable {\tt f} has the static type {\tt (-> String String)}
 then the application {\tt (f x)} must be rewritten to check for a string result
 because {\tt f} could be an untyped function.
Rewriting also inserts hooks to update a global blame heap.
When a value crosses a boundary, the blame heap records a cast assumption.
Elimination forms and other operations record smaller path assumptions.
If a transient check ever fails, the language follows path assumptions and
 reports a set of casts/boundaries to the programmer.



 
%% -----------------------------------------------------------------------------

As section~\ref{sec:challenges} explains, the first challenge is to
generate a representative mistakes in representative software. The
following two subsections explain in reverse order. 

%% -----------------------------------------------------------------------------
\def\sub#1#2{\subsection{#2} \label{sub:mutate-#1} \input{mutate-#1.tex}}

\sub{benchmarks} {Which Software to Use As Benchmarks}
\sub{mutators}   {How to Mutate the Benchmarks} 
\sub{interesting}{Are These Mutators Interesting} 

%% -----------------------------------------------------------------------------

Section~\ref{sec:why-rational} explains, and illustrates with an example, how a
migratory type setting helps find the source of an impedance mismatch. Roughly
speaking, it encourages the rational programmer equip a module with types if it
is blamed in an error message. A sequence of such steps make up a portion of the
migratory path in the lattice of type migration~\cite{tfgnvf-popl-2016} (see
sec.~\ref{sub:stuff} for details). 

Unfortunately, the rational programmer receives different kinds of information
and thus may construct different paths. As section~\ref{sec:landscape} shows,
the three different semantics deal with the same program in three different
ways, and the error messages point to different sources of a problem.

Hence the research problem is how to view the actions of a rational programmer
comparable across all semantics. Sections~\ref{sub:natural}
through~\ref{sub:erasure} describe the various rational programmers and explain
to what extent their actions are comparable. Section\ref{sub:experiment}
enhances comparability with the programmer-effort dimension. Finally, the last
subsection can then state the experimental questions. 

%% -----------------------------------------------------------------------------
\def\rsub#1#2{\subsection{#2} \label{sub:#1} \input{rational-#1.tex}}

\rsub{stuff}     {The Lattice and the Debugging Scenario} 
\rsub{natural}   {The Natural Rational Programmer} 
\rsub{transient} {The Transient Rational Programmer} 
\rsub{erasure}   {The Erasure Rational Programmer} 
\rsub{effort}    {The Programmer Effort} \label{subsec:effort}
\rsub{experiment}{The Experimental Questions} 
	 \label{sec:rational}
\section{How to Sample the Experimental Space} 

The mutators from section~\ref{sec:mutate} generate 96,515,744 debugging
scenarios; far too many to even identify the interesting ones among them.
Furthermore this population is heterogeneous; different scenarios come
from different mutants and different mutants are the result of a
different mutator and benchmark pair. Hence, to make the experiment
computationally feasible and statistically precise, we perform stratified
uniform random sampling with three levels of stratification.  The first
two strata group mutants first by benchmark and then by mutator. 
We randomly select from the mutants of
each benchmark ensuring that, per
benchmark, the sample reflects the diversity of mutants with respect to
the mutators that generated them.  Specifically,  we sample 80 mutants
with interesting scenarios per benchmark,  evenly-distributed across all
of the mutators that contribute mutants for the benchmark.  The third
stratum of our sampling is the lattice of a mutant and we draw 96 random
interesting scenarios from each lattice. For lattices that have less than
96 interesting scenarios, we select them all.

In summary, our sampling process takes advantage of ancillary information
to deliver a  sufficiently large representative set of scenarios.
Concretely, we test each
mode of the rational programmer on 72,192 randomly selected interesting
debugging scenarios spread across both benchmarks and mutators.
Furthermore stratified random sampling allows us to derive precise conclusions
about the proportion of the whole population that satisfies a given trait 
from the proportion of the sample 
that satisfies the trait despite the population's heterogeneity, exactly what we need to project comparisons about the
usefulness of various modes of the
rational programmer from the sample to the set of all interesting debugging
scenarios.  Finally, we can also calculate separate estimates per stratum
and provide an alternative
viewpoint of our results. For instance, because of the size of the sample
of scenarios from the lattice of a mutant, we can estimate whether natural
blame is more useful than natural exceptions for all interesting scenarios
of each mutant with 95\% confidence and 5\% error margin.

 


%% we use the following algorithm to select the 80 mutants evenly distributed across all of the mutators that do produce mutants of a benchmark:
%% 1. Treat each mutator as a bucket containing the mutants produced by that mutator
%% 2. Let the target-sample-size be 80
%% 3. Let the bucket-sample-size be target-sample-size/number-of-buckets
%% 4. For each bucket, do
%%    1. Randomly remove min(bucket-sample-size, mutants-in(bucket)) mutants from the bucket
%%    2. Set target-sample-size = target-sample-size - (the number of mutants removed)
%% 5. If target-sample-size is empty, done; else, go back to 3.


%% -----------------------------------------------------------------------------


The test bed for executing the experimental process utilizes a machine with two
Intel Xeon Gold 6258R processors (28 doubly-threaded cores each) and 500GB of
memory.  Each debugging scenario had a 4 minute timeout and a 6GB memory
limit. Running the experiment on all debugging scenarios took over
30,000 compute hours or roughly three-and-a-half compute years. 

\
\begin{figure} \footnotesize 
  \centering
  \includegraphics[width=\textwidth]{./plots/avo-bars}

  \vspace{1em}
  \begin{minipage}{0.95\textwidth}
      Each plot depicts a head-to-head comparison of the mode named above the
      plot vs. every other mode.  The (green) portion above 0 is the estimated
      percentage of scenarios where the named mode is more useful than the
      other.  The (red) portion below 0 is the estimated percentage of scenarios
      where the named mode is less useful than the other.  The upper bound
      margin of error is 0.02\%.
  \end{minipage}

  \caption{Usefulness comparisons}
  \label{fig:avo-bars}
\end{figure}

\begin{figure}\footnotesize 
  \centering
  \includegraphics[width=0.32\textwidth]{./plots/TR-TR-stack-first-venn}
  \hfill
  \includegraphics[width=0.32\textwidth]{./plots/transient-newest-transient-stack-first-venn}
  \hfill
  \includegraphics[width=0.32\textwidth]{./plots/transient-oldest-transient-stack-first-venn}

  \vspace{1em}
  \begin{minipage}{0.95\textwidth}

     Each diagram shows the overlap of the successful scenarios for three modes.
     For example, in the leftmost diagram, all three modes succeed on the same
     scenario 57.3\% of the time, only Natural and Natural exceptions succeed on
     29.1\% of the scenarios, only Natural and Erasure succeed on 2.1\%, and
     Natural alone succeeds on 9\%. The upper bound margin of error is 0.02\%.

  \end{minipage}

  \caption{Blame usefulness analysis}
  \label{fig:success-venns}
\end{figure}


% \footnote{The random mode behaves the same for all three semantics, so this analysis considers it a single mode.}

Figure~\ref{fig:success-bars} summarizes the overall success rates of
every mode.
The success rates illustrate a few points that underlie the
rest of the analysis.  The first notable piece of information from this
figure is that every mode has failed debugging scenarios, not just
Erasure. This should not come as a surprise to the astute reader.  Running
a rational programmer mode on a scenario may result in an exception
that carries no useful information about which module to equip with types next. For instance the stack trace of the exception may not
contain frames from any untyped module of the program. This can happen at any point
along a blame trail, causing it to fail.

\begin{wrapfigure}{r}{.40\textwidth} \footnotesize 
  \vspace{-1.5em}
  \includegraphics[width=0.38\textwidth]{./plots/success-bars}
  \vspace{1em}
  \begin{minipage}{0.35\textwidth} \raggedright
%   The estimated percentage of scenarios for which each mode succeeds in locating the bug.
   The upper bound margin of error is 0.02\%.
  \end{minipage}
\vspace{-2em}
  \caption{Percentage rates of success} \label{fig:success-bars}
\end{wrapfigure}



While most blame trail failures follow the above pattern, a few do
not.  Breaking down the failure reasons for Natural blame (1748 in
total) reveals an additional cause. For a small set of debugging scenarios
(40), Natural produces a run-time type error blaming a non-buggy
already-typed module. All these cases are due to known open issues with Typed
Racket and class contracts. 

In Transient, similar to Natural, most failures are due to unhelpful exception
information (1851 for both Transient first and last blame).  However, Transient
also has a substantial number of failures because scenarios hit the time and/or
memory limits of the experiment (\textasciitilde770 scenarios).  Additionally,
there are nearly 1,000 cases where Transient reports an empty blame set, leaving
the rational programmer without hints about how to proceed.
Sections~\ref{sec:threat:transient} and ~\ref{sec:threat:transient2} address
these causes of failure for Transient and how they affect the experiment.

The second key observation from figure~\ref{fig:success-bars} is that the modes
that use blame all outperform those that do not. In particular, Natural and both
of Transient's blame modes succeed in 85 - 90\% of the scenarios, while their
corresponding exception modes succeed in less than 80\% of them, and so too for
Erasure. The only exception is that the random programmer always succeeds; the
figure omits this mode because it just reflects the fact that every scenario has
finitely many modules, so the random programmer eventually types the buggy
module.

Figure~\ref{fig:avo-bars} depicts a head-to-head comparison of every
mode's performance against every other mode (except Random). The
comparison  answers the four questions from section~\ref{sub:experiment}. 
Each plot shows the proportion of scenarios where one mode performs
better or worse than each other mode.  In particular, each bar above zero
represents the proportion where the plot's named mode succeeds and the
mode on the x-axis fails; the corresponding bar below zero represents the
proportion of the inverse case.  For example, the plot titled ``Natural''
shows that Natural outperforms Natural exceptions in about 11\% of the
scenarios, and the inverse (Natural performs worse than Natural
exceptions) never happens.  Similarly, the plot titled ``Transient last
blame'' shows that Transient last blame outperforms Natural exceptions
 in about 9\% of the scenarios, but conversely it performs worse
than Natural exceptions in about 2\% of the scenarios.

The figure answers questions $Q_1$, $Q_2$ and $Q_3$ affirmatively.  
In all three semantics, blame modes outperform their
corresponding exception mode by \textasciitilde10\%.  The
Natural exceptions mode is never more useful than Natural blame, and
Transient exceptions are more useful than Transient first and Transient
last blame in a small percentage (less than 1\%) of the scenarios. 

Figure~\ref{fig:avo-bars} also provides answers for $Q_*$. Blame for all three
semantics is significantly more useful than Erasure exceptions---by almost
12\% for Natural and almost 9\% for Transient. Natural blame is more useful than
both versions of Transient blame by a small percentage (about 4\%). The
Transient first and Transient last blame are practically indistinguishable.
Finally, Natural exceptions are more useful than Transient exceptions, although
only in a small percentage of scenarios (about 2.5\%). A rare few scenarios
(about 0.5\%) show the opposite, despite the theoretically advantageous
additional checks of Natural.

\begin{figure} \footnotesize \centering
  \includegraphics[width=\textwidth]{./plots/bt-lengths-table}

  \vspace{1em}
  \begin{minipage}{0.95\textwidth}
    Each plot depicts the distribution of trail lengths for a given mode across all benchmarks.
    The upper bound margin of error is 0.05\%.
  \end{minipage}

  \caption{Programmer effort} \label{fig:effort-table}
\end{figure}

\begin{figure}\footnotesize \centering
  \includegraphics[width=\textwidth]{./plots/bt-length-comparisons}

  \vspace{1em}
  \begin{minipage}{0.95\textwidth}

    Each plot depicts the distribution of scenarios with trail length
    differences ranging from -3 to 3. A $-x$ difference denotes that the first
    mode's trail is $x$ steps {\em shorter\/} than the second mode's trail for the
    same scenario; a positive difference denotes the inverse. A difference of
    $\infty$ indicates one mode's trail succeeds while other mode's fails. The
    15|60 on the y-axis indicates that the axis is truncated between 15 and
    60\%.
    The upper bound margin of error is 0.03\%.

  \end{minipage}

  \caption{Effort comparisons} \label{fig:effort-comparisons}
\end{figure}

An alternative way to understand the answers for questions $Q_1$ to $Q_3$, is to
analyze the success of each semantics in comparison to
Erasure. Figure~\ref{fig:success-venns} depicts the results of this analysis.
Specifically, the figure shows one Venn diagram per mode of the rational
programmer that uses blame.  Each diagram shows the overlap of successful
scenarios for the blame mode, its corresponding exception mode, and Erasure.
For example, the leftmost diagram (Natural) shows that all three modes succeed
on 75.7\% of the scenarios, only Natural and Natural exceptions succeed on
11.6\% of the scenarios, only Natural and Erasure succeed on 1.8\%, and Natural
alone succeeds on 9.2\%.  This analysis highlights the success trade-offs each
semantics offers against Erasure, with and without blame. For instance, the
analysis for Natural clearly illustrates that, when choosing between Natural
blame, Natural exceptions, and Erasure, Natural blame is the absolutely most
successful: all of the successes of the other two modes are subsets of Natural's
successful scenarios.  On the other hand, Transient's blame modes fare similarly
but the choice is not so clear-cut.

Turning to programmer effort, figure~\ref{fig:effort-table} shows the
estimated distribution of blame trail lengths for the interesting debugging
scenarios. There are two immediate take-aways from the figure. First, the effort for
successfully debugging interesting scenarios (in green) for the random
mode of the rational programmer is highly spread out, as
expected. In contrast, in the other modes, successful effort coalesces at
the left side of the plot, meaning that in most cases the programmer needs
to type a single module to debug a scenario. 

Second, the exceptions of the Erasure semantics either help the rational
programmer immediately or the rational programmer fails to debug a scenario
altogether (in red).  This is expected; adding type annotations to an Erasure
program does not change its behavior, so if the type checker does not reject the
program with the new annotations, the rational programmer is stuck.  In other
words, an exception from the run-time system has to point to the buggy module with the
first try. Otherwise the rational programmer types an irrelevant module, runs
the program again, and the same exception points again to the already typed
module.

Figure~\ref{fig:effort-comparisons} provides head-to-head comparisons of effort.
The comparison between two modes boils down to the difference in length between
their trails for all scenarios where they both succeed. Hence, each plot in the
figure shows the distribution of scenarios with length differences ranging from
-3 (the first mode's trail is 3 steps shorter than the second's) to 3 (the first
mode's trail is 3 steps longer than the second's). The figure offers several
insights about how modes compare in terms of effort that complement the insights
about how they compare in terms of success rates from figure~\ref{fig:avo-bars}.
First, Natural blame never produces shorter trails than Natural exceptions, and
sometimes (albeit rarely) produces slightly longer ones. Hence, the experiment
provides evidence that blame helps the rational programmer debug more scenarios
but does not shorten the debugging process compared to exceptions. Second,
Natural relatively often (close to 8\% of the scenarios) produces shorter trails
than both Transient blame modes, and sometimes the trails are significantly
shorter. Finally, Transient's blame modes also share the characteristic with
Natural that blame sometimes produces longer trails than their corresponding
exception modes.

      \label{sec:results}
\section{What We Learned}

\begin{itemize}
  \item Interesting discoveries...?

  \item Comparison of existing GT systems represented in our study.  
    Natural is here, transient is here, erasure is here, and they
        compare as...

      \item Threats to validity:
        \begin{itemize}
          \item Racket stack traces
          \item Mutant sampling
          \item Blame trails do not necessarily capture all aspects of error reporting
          \item Transient blame adaptations
          \item More?
        \end{itemize}
\end{itemize}        



   \label{sec:discussion}
%% -----------------------------------------------------------------------------

\citet{lksfd-popl-2020}'s human-out-of-the-loop empirical analysis of blame for
higher-order contracts in the inspiration for this paper.  While they do not
spell out the notion of the rational programmer, they present the basic
ingredients of the technique.

Technically, this work and~\citet{lksfd-popl-2020}'s differ in many ways. The
experiment presented here concerns three different modes of gradual typing,
theirs a single notion of contract checking.  It also contributes the idea of
creating three comparable rational programmers (with several modes). The
type-level mutators represent another technical contribution of the work
presented here; as explained in section~\ref{sec:mutate}, almost none
of~\citet{lksfd-popl-2020}'s mutators is useful at the level of type mistakes.

The literature on gradual typing presents many semantics beyond the three used
here and only two additional blame strategies.  Pyret
(\shorturl{https://www.}{pyret.org}) assigns fixed-size data types the Natural
semantics and functions a Transient semantics. The Forgetful~\cite{cl-icfp-2017}
and the Amnesic~\cite{gfd-oopsla-2019} semantics are similar to Transient but
use wrappers instead of in-lined checks.  Nom~\cite{mt-oopsla-2017} and other
\emph{concrete\/} semantics~\cite{wnlov-popl-2010, rsfbv-popl-2015,
rzv-ecoop-2015, rat-oopsla-2017} assume that every value comes with a type tag
and use tag checks to supervise the interactions between typed and untyped code.
Monotonic references~\cite{svctg-esop-2015} and the
semantics~\cite{tlt-popl-2019, etg-icfp-19, tt-scp-20, tgt-popl-18, tt-sas-17}
derived with the Abstracting Gradual Typing technique~\cite{gct-popl-2016} are
variants of Natural.  Among these representative semantics, only Amnesic and Nom
present innovative blame strategies.  The experiment excludes these strategies
because they seem to be merely theoretical constructions and/or impose severe
restrictions on programmers.
 
      \label{sec:related}
\section{What to Do Next}


   \label{sec:conclusion}

%% Acknowledgments
%%\begin{acks}                 
  %% acks environment is optional
  %% contents suppressed with 'anonymous'
  %% Commands \grantsponsor{<sponsorID>}{<name>}{<url>} and
  %% \grantnum[<url>]{<sponsorID>}{<number>} should be used to
  %% acknowledge financial support and will be used by metadata
  %% extraction tools.
%%\end{acks}


%% Bibliography
\bibliography{cs}

\end{document}
