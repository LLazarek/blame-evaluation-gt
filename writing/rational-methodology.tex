
Each time the rational programmer debugs a program, it is
validation for programming languages researchers; their blame assignment 
mechanisms provide information that points towards the source of the
problem. Moreover, their slogans and theorems help programmers connect the dots.
In contrast, when the rational programmer fails, it speaks for the
limited predictive power of programming languages theory and the
misleading projections of language design.

In other words the rational programmer implies a methodology for
evaluating
the design of programming languages features such as blame assignment mechanisms. 
In general terms, the methodology consists of four 
pieces:

\begin{itemize} 

\item The description of a  rational programmer as an algorithmic
  procedure that examines a program and performs an action. 
    Typically the action of the rational program is consistent 
   with the theory of the features under examination.

  \item A hypothesis about whether a sequence of actions of the
    rational programmer leads to a desired outcome. 
 
\item A large-scale experiment that tests the hypothesis 
  on a large number of representative scenarios. 


\item The evaluation of the features based on the outcome of the
  experiment.    

\end{itemize}

In specific terms for blame assignment mechanisms, the methodology
imagines:  

\begin{itemize}
\item The rational programmer as a procedure that consumes a program and the error message 
  the evaluation of the program produces and, uses the error message to
    select the next component of the program to add types.

  \item The rational programmer hypothesis as  research question whether 
    for all programs with
    type-level mistakes a series of steps of the rational programmer always leads to a type
    error that points to the component that contains the mistake. 

\item The experiment as the attempts of the rational programmer to debug a
  large number of programs with type-level mistakes. 


\item The evaluation of a blame assignment mechanism based on whether the
  information it provides is critical for the outcome of the experiment.   
\end{itemize}




